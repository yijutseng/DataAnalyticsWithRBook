\documentclass[]{book}
\usepackage{lmodern}
\usepackage{amssymb,amsmath}
\usepackage{ifxetex,ifluatex}
\usepackage{fixltx2e} % provides \textsubscript
\ifnum 0\ifxetex 1\fi\ifluatex 1\fi=0 % if pdftex
  \usepackage[T1]{fontenc}
  \usepackage[utf8]{inputenc}
\else % if luatex or xelatex
  \ifxetex
    \usepackage{mathspec}
  \else
    \usepackage{fontspec}
  \fi
  \defaultfontfeatures{Ligatures=TeX,Scale=MatchLowercase}
\fi
% use upquote if available, for straight quotes in verbatim environments
\IfFileExists{upquote.sty}{\usepackage{upquote}}{}
% use microtype if available
\IfFileExists{microtype.sty}{%
\usepackage{microtype}
\UseMicrotypeSet[protrusion]{basicmath} % disable protrusion for tt fonts
}{}
\usepackage[margin=1in]{geometry}
\usepackage{hyperref}
\hypersetup{unicode=true,
            pdftitle={資料科學與R語言},
            pdfauthor={曾意儒 Yi-Ju Tseng},
            pdfborder={0 0 0},
            breaklinks=true}
\urlstyle{same}  % don't use monospace font for urls
\usepackage{natbib}
\bibliographystyle{apalike}
\usepackage{color}
\usepackage{fancyvrb}
\newcommand{\VerbBar}{|}
\newcommand{\VERB}{\Verb[commandchars=\\\{\}]}
\DefineVerbatimEnvironment{Highlighting}{Verbatim}{commandchars=\\\{\}}
% Add ',fontsize=\small' for more characters per line
\usepackage{framed}
\definecolor{shadecolor}{RGB}{248,248,248}
\newenvironment{Shaded}{\begin{snugshade}}{\end{snugshade}}
\newcommand{\KeywordTok}[1]{\textcolor[rgb]{0.13,0.29,0.53}{\textbf{{#1}}}}
\newcommand{\DataTypeTok}[1]{\textcolor[rgb]{0.13,0.29,0.53}{{#1}}}
\newcommand{\DecValTok}[1]{\textcolor[rgb]{0.00,0.00,0.81}{{#1}}}
\newcommand{\BaseNTok}[1]{\textcolor[rgb]{0.00,0.00,0.81}{{#1}}}
\newcommand{\FloatTok}[1]{\textcolor[rgb]{0.00,0.00,0.81}{{#1}}}
\newcommand{\ConstantTok}[1]{\textcolor[rgb]{0.00,0.00,0.00}{{#1}}}
\newcommand{\CharTok}[1]{\textcolor[rgb]{0.31,0.60,0.02}{{#1}}}
\newcommand{\SpecialCharTok}[1]{\textcolor[rgb]{0.00,0.00,0.00}{{#1}}}
\newcommand{\StringTok}[1]{\textcolor[rgb]{0.31,0.60,0.02}{{#1}}}
\newcommand{\VerbatimStringTok}[1]{\textcolor[rgb]{0.31,0.60,0.02}{{#1}}}
\newcommand{\SpecialStringTok}[1]{\textcolor[rgb]{0.31,0.60,0.02}{{#1}}}
\newcommand{\ImportTok}[1]{{#1}}
\newcommand{\CommentTok}[1]{\textcolor[rgb]{0.56,0.35,0.01}{\textit{{#1}}}}
\newcommand{\DocumentationTok}[1]{\textcolor[rgb]{0.56,0.35,0.01}{\textbf{\textit{{#1}}}}}
\newcommand{\AnnotationTok}[1]{\textcolor[rgb]{0.56,0.35,0.01}{\textbf{\textit{{#1}}}}}
\newcommand{\CommentVarTok}[1]{\textcolor[rgb]{0.56,0.35,0.01}{\textbf{\textit{{#1}}}}}
\newcommand{\OtherTok}[1]{\textcolor[rgb]{0.56,0.35,0.01}{{#1}}}
\newcommand{\FunctionTok}[1]{\textcolor[rgb]{0.00,0.00,0.00}{{#1}}}
\newcommand{\VariableTok}[1]{\textcolor[rgb]{0.00,0.00,0.00}{{#1}}}
\newcommand{\ControlFlowTok}[1]{\textcolor[rgb]{0.13,0.29,0.53}{\textbf{{#1}}}}
\newcommand{\OperatorTok}[1]{\textcolor[rgb]{0.81,0.36,0.00}{\textbf{{#1}}}}
\newcommand{\BuiltInTok}[1]{{#1}}
\newcommand{\ExtensionTok}[1]{{#1}}
\newcommand{\PreprocessorTok}[1]{\textcolor[rgb]{0.56,0.35,0.01}{\textit{{#1}}}}
\newcommand{\AttributeTok}[1]{\textcolor[rgb]{0.77,0.63,0.00}{{#1}}}
\newcommand{\RegionMarkerTok}[1]{{#1}}
\newcommand{\InformationTok}[1]{\textcolor[rgb]{0.56,0.35,0.01}{\textbf{\textit{{#1}}}}}
\newcommand{\WarningTok}[1]{\textcolor[rgb]{0.56,0.35,0.01}{\textbf{\textit{{#1}}}}}
\newcommand{\AlertTok}[1]{\textcolor[rgb]{0.94,0.16,0.16}{{#1}}}
\newcommand{\ErrorTok}[1]{\textcolor[rgb]{0.64,0.00,0.00}{\textbf{{#1}}}}
\newcommand{\NormalTok}[1]{{#1}}
\usepackage{longtable,booktabs}
\usepackage{graphicx,grffile}
\makeatletter
\def\maxwidth{\ifdim\Gin@nat@width>\linewidth\linewidth\else\Gin@nat@width\fi}
\def\maxheight{\ifdim\Gin@nat@height>\textheight\textheight\else\Gin@nat@height\fi}
\makeatother
% Scale images if necessary, so that they will not overflow the page
% margins by default, and it is still possible to overwrite the defaults
% using explicit options in \includegraphics[width, height, ...]{}
\setkeys{Gin}{width=\maxwidth,height=\maxheight,keepaspectratio}
\IfFileExists{parskip.sty}{%
\usepackage{parskip}
}{% else
\setlength{\parindent}{0pt}
\setlength{\parskip}{6pt plus 2pt minus 1pt}
}
\setlength{\emergencystretch}{3em}  % prevent overfull lines
\providecommand{\tightlist}{%
  \setlength{\itemsep}{0pt}\setlength{\parskip}{0pt}}
\setcounter{secnumdepth}{5}
% Redefines (sub)paragraphs to behave more like sections
\ifx\paragraph\undefined\else
\let\oldparagraph\paragraph
\renewcommand{\paragraph}[1]{\oldparagraph{#1}\mbox{}}
\fi
\ifx\subparagraph\undefined\else
\let\oldsubparagraph\subparagraph
\renewcommand{\subparagraph}[1]{\oldsubparagraph{#1}\mbox{}}
\fi

%%% Use protect on footnotes to avoid problems with footnotes in titles
\let\rmarkdownfootnote\footnote%
\def\footnote{\protect\rmarkdownfootnote}

%%% Change title format to be more compact
\usepackage{titling}

% Create subtitle command for use in maketitle
\newcommand{\subtitle}[1]{
  \posttitle{
    \begin{center}\large#1\end{center}
    }
}

\setlength{\droptitle}{-2em}
  \title{資料科學與R語言}
  \pretitle{\vspace{\droptitle}\centering\huge}
  \posttitle{\par}
  \author{曾意儒 Yi-Ju Tseng}
  \preauthor{\centering\large\emph}
  \postauthor{\par}
  \predate{\centering\large\emph}
  \postdate{\par}
  \date{2017-02-07}

\usepackage{booktabs}
\usepackage{amsthm}

\usepackage{CJKutf8}

\makeatletter
\def\thm@space@setup{%
  \thm@preskip=8pt plus 2pt minus 4pt
  \thm@postskip=\thm@preskip
}
\makeatother

\usepackage{amsthm}
\newtheorem{theorem}{Theorem}[chapter]
\newtheorem{lemma}{Lemma}[chapter]
\theoremstyle{definition}
\newtheorem{definition}{Definition}[chapter]
\newtheorem{corollary}{Corollary}[chapter]
\newtheorem{proposition}{Proposition}[chapter]
\theoremstyle{definition}
\newtheorem{example}{Example}[chapter]
\theoremstyle{remark}
\newtheorem*{remark}{Remark}
\begin{document}
\maketitle

{
\setcounter{tocdepth}{1}
\tableofcontents
}
\chapter*{}\label{preface}
\addcontentsline{toc}{chapter}{}

\href{http://im.cgu.edu.tw/bin/home.php}{長庚大學資訊管理學系}
\href{https://github.com/yijutseng/BigDataCGUIM}{大數據分析方法}教學使用書籍,內容包括使用\href{http://www.r-project.org/}{R語言}做資料擷取、資料清洗與處理、探索式資料分析、資料視覺化與資料探勘等,並介紹R與Hadoop
EcoSystems介接方法。

\chapter{R語言101}\label{intro}

本章節介紹學習R語言的基本知識,包括基本指令操作、運算子介紹等。

\section{什麼是R語言}\label{r}

\href{http://www.r-project.org/}{R語言}是一種自由軟體程式語言,主要用於資料分析與統計運算,2000年時終於發表R
1.0.0,有關R語言的發展歷史可參考\href{https://zh.wikipedia.org/wiki/R\%E8\%AF\%AD\%E8\%A8\%80}{維基百科}。基本的R軟體已經內建多種統計及分析功能,其餘功能可以透過安裝\textbf{套件(Packages)}加載,眾多的套件使R的使用者可以【站在巨人的肩膀上(Standing
on the shoulders of giants (Hal R. Varian,
Google))】做資料分析,截至2017年1月為止,R軟體可另外安裝的套件數目共有10,000個以上
(\href{https://www.rstudio.com/rviews/2017/01/06/10000-cran-packages/}{R
Studio報導})。常用的套件清單可參考各項網路資訊,如\href{https://support.rstudio.com/hc/en-us/articles/201057987-Quick-list-of-useful-R-packages}{R
Studio的整理:Quick list of useful R packages}

安裝套件Package的方法如下:

\begin{Shaded}
\begin{Highlighting}[]
\KeywordTok{install.packages}\NormalTok{(}\StringTok{"套件名稱"}\NormalTok{)}
\end{Highlighting}
\end{Shaded}

值得注意的是,套件名稱需要加上雙引號,舉例來說,若要安裝\texttt{ggplot2}套件,則要在R的Console視窗內輸入:

\begin{Shaded}
\begin{Highlighting}[]
\KeywordTok{install.packages}\NormalTok{(}\StringTok{"ggplot2"}\NormalTok{)}
\end{Highlighting}
\end{Shaded}

若要載入\textbf{已安裝}的套件,則輸入\texttt{library(套件名稱)},範例:

\begin{Shaded}
\begin{Highlighting}[]
\KeywordTok{library}\NormalTok{(ggplot2)}
\end{Highlighting}
\end{Shaded}

載入已安裝的套件時,\textbf{不用}在套件名稱前後加雙引號。

\section{函數使用}

在R中有許多內建函數,安裝套件後各套件也會提供各式各樣寫好的函數,函數使用方式為\texttt{函數名稱(參數1,參數2,....)},以計算平均數為例,可使用\texttt{mean()}函數,範例如下:

\begin{Shaded}
\begin{Highlighting}[]
\KeywordTok{mean}\NormalTok{(}\KeywordTok{c}\NormalTok{(}\DecValTok{1}\NormalTok{,}\DecValTok{2}\NormalTok{,}\DecValTok{3}\NormalTok{,}\DecValTok{4}\NormalTok{,}\DecValTok{5}\NormalTok{,}\DecValTok{6}\NormalTok{)) ##計算1~6的平均數}
\end{Highlighting}
\end{Shaded}

\begin{verbatim}
## [1] 3.5
\end{verbatim}

若想知道各函數所需參數,可使用\texttt{?函數名稱}觀看函數作者所撰寫的說明文件

\begin{Shaded}
\begin{Highlighting}[]
\NormalTok{?mean}
\end{Highlighting}
\end{Shaded}

除非有指定參數名稱,函數的參數設定有順序性,如序列產生函數\texttt{seq()},參數順序為\texttt{from,\ to,\ by},代表序列起點、序列終點,以及相隔單位。

\begin{Shaded}
\begin{Highlighting}[]
\KeywordTok{seq}\NormalTok{(}\DataTypeTok{from=}\DecValTok{1}\NormalTok{,}\DataTypeTok{to=}\DecValTok{9}\NormalTok{,}\DataTypeTok{by=}\DecValTok{2}\NormalTok{)}\CommentTok{#1~9,每隔2產生一數字}
\end{Highlighting}
\end{Shaded}

\begin{verbatim}
## [1] 1 3 5 7 9
\end{verbatim}

\begin{Shaded}
\begin{Highlighting}[]
\KeywordTok{seq}\NormalTok{(}\DecValTok{1}\NormalTok{,}\DecValTok{9}\NormalTok{,}\DecValTok{2}\NormalTok{)}\CommentTok{#按照順序輸入參數,可省去參數名稱}
\end{Highlighting}
\end{Shaded}

\begin{verbatim}
## [1] 1 3 5 7 9
\end{verbatim}

\begin{Shaded}
\begin{Highlighting}[]
\KeywordTok{seq}\NormalTok{(}\DataTypeTok{by=}\DecValTok{2}\NormalTok{,}\DataTypeTok{to=}\DecValTok{9}\NormalTok{,}\DataTypeTok{from=}\DecValTok{1}\NormalTok{)}\CommentTok{#若不想照順序輸入參數,需要指定參數名稱}
\end{Highlighting}
\end{Shaded}

\begin{verbatim}
## [1] 1 3 5 7 9
\end{verbatim}

\section{變數設定}

在開始深入學習R語言之前,首要任務是學習最基本的R程式碼:\textbf{變數設定},在R語言中,主要使用\texttt{\textless{}-}設定變數,設定方法為:\textbf{變數名稱}\texttt{\textless{}-}\textbf{變數內容(值)},雖然\textbf{變數名稱}可依箭頭方向放置於左側\texttt{\textless{}-}或右側\texttt{-\textgreater{}},但為方便閱讀,\textbf{變數名稱}多放置於左側。

\begin{Shaded}
\begin{Highlighting}[]
\NormalTok{a<-}\DecValTok{1} 
\DecValTok{2}\NormalTok{->b}
\NormalTok{a}
\end{Highlighting}
\end{Shaded}

\begin{verbatim}
## [1] 1
\end{verbatim}

\begin{Shaded}
\begin{Highlighting}[]
\NormalTok{b}
\end{Highlighting}
\end{Shaded}

\begin{verbatim}
## [1] 2
\end{verbatim}

R語言也接受使用\texttt{=}設定變數,此時\textbf{變數名稱}必須在左側,如:\textbf{變數名稱}\texttt{=}\textbf{變數內容}

\begin{Shaded}
\begin{Highlighting}[]
\NormalTok{c=}\DecValTok{1} 
\NormalTok{c}
\end{Highlighting}
\end{Shaded}

\begin{verbatim}
## [1] 1
\end{verbatim}

除了\textbf{變數設定}外,\texttt{str()}函數也為常用基本函數,\texttt{str()}用在檢查與總覽各類變數型態。

\begin{Shaded}
\begin{Highlighting}[]
\NormalTok{d<-}\DecValTok{3}
\KeywordTok{str}\NormalTok{(d)}
\end{Highlighting}
\end{Shaded}

\begin{verbatim}
##  num 3
\end{verbatim}

變數的命名有以下規則:

\begin{itemize}
\tightlist
\item
  不可使用保留字,如break, else, FALSE, for, function, if, Inf, NA, NaN,
  next, repeat, return, TRUE, while等
\item
  開頭只能是英文字,或 \texttt{.}
\item
  大小寫敏感
\end{itemize}

\section{執行視窗}

R是可直譯的語言,也就是說,可以在執行視窗(Console)直接打程式碼,在視窗出現\texttt{\textgreater{}}時,表示可輸入指令,若視窗出現\texttt{+}時,表示前面的程式碼還沒打完,必須鍵入完整的程式碼讓R執行。

\section{資料型態}\label{DataType}

在R語言中,常用的資料型態包括\textbf{數值 (numeric)}、\textbf{字串
(character)}、\textbf{布林變數 (logic)}以及\textbf{日期 (Date)}等。

\subsection{數值 numeric}\label{-numeric}

數值包括整數(沒有小數點)與符點數(有小數點)的數值

\begin{Shaded}
\begin{Highlighting}[]
\NormalTok{num1<-}\DecValTok{100} 
\NormalTok{num2<-}\FloatTok{1000.001}
\end{Highlighting}
\end{Shaded}

值得注意的是,若數值長度超過 \texttt{2\^{}53},必須導入\texttt{bit64}
package
\citep{R-bit64},將數值長度上限提高為\texttt{2\^{}63},才能表示完整數值

\begin{Shaded}
\begin{Highlighting}[]
\KeywordTok{print}\NormalTok{(}\DecValTok{2}\NormalTok{^}\DecValTok{53}\NormalTok{, }\DataTypeTok{digits=}\DecValTok{20}\NormalTok{) }
\end{Highlighting}
\end{Shaded}

\begin{verbatim}
## [1] 9007199254740992
\end{verbatim}

\begin{Shaded}
\begin{Highlighting}[]
\KeywordTok{print}\NormalTok{(}\DecValTok{2}\NormalTok{^}\DecValTok{53+1}\NormalTok{, }\DataTypeTok{digits=}\DecValTok{20}\NormalTok{) }\CommentTok{# +1後,數值仍與2^53相同}
\end{Highlighting}
\end{Shaded}

\begin{verbatim}
## [1] 9007199254740992
\end{verbatim}

\begin{Shaded}
\begin{Highlighting}[]
\KeywordTok{library}\NormalTok{(bit64) }\CommentTok{# 導入bit64 package}
\KeywordTok{print}\NormalTok{(}\KeywordTok{as.integer64}\NormalTok{(}\DecValTok{2}\NormalTok{)^}\DecValTok{53}\NormalTok{, }\DataTypeTok{digits=}\DecValTok{20}\NormalTok{)}
\end{Highlighting}
\end{Shaded}

\begin{verbatim}
## integer64
## [1] 9007199254740992
\end{verbatim}

\begin{Shaded}
\begin{Highlighting}[]
\KeywordTok{print}\NormalTok{(}\KeywordTok{as.integer64}\NormalTok{(}\DecValTok{2}\NormalTok{)^}\DecValTok{53+1}\NormalTok{, }\DataTypeTok{digits=}\DecValTok{20}\NormalTok{)}\CommentTok{# 導入bit64後,可得正確答案}
\end{Highlighting}
\end{Shaded}

\begin{verbatim}
## integer64
## [1] 9007199254740993
\end{verbatim}

\subsection{字串 character}\label{-character}

用雙引號\texttt{"}框起的文字會被儲存為字串格式,若在數字前後加上雙引號,數字也會被儲存為文字形式,無法進行數值的加減乘除等運算。

\begin{Shaded}
\begin{Highlighting}[]
\NormalTok{char1<-}\StringTok{"abcTest"} 
\NormalTok{char2<-}\StringTok{"100"}
\NormalTok{char3<-}\StringTok{"200"}
\CommentTok{#char2+char3 #會輸出Error message: non-numeric argument to binary operator}
\end{Highlighting}
\end{Shaded}

\subsection{布林變數 logic}\label{-logic}

用於邏輯判斷,可使用大寫\textbf{TRUE}或\textbf{T}代表\textbf{真},大寫\textbf{FALSE}或\textbf{F}代表假。

\begin{Shaded}
\begin{Highlighting}[]
\NormalTok{boolT<-}\OtherTok{TRUE}
\NormalTok{boolT1<-T}
\NormalTok{boolF<-}\OtherTok{FALSE}
\NormalTok{boolF1<-F}
\end{Highlighting}
\end{Shaded}

\subsection{日期 (Date)}\label{-date}

用於表示日期,於資料分析中常用,使用\texttt{Sys.Date()}指定可得系統日期。

\begin{Shaded}
\begin{Highlighting}[]
\NormalTok{dateBook<-}\KeywordTok{Sys.Date}\NormalTok{()}
\NormalTok{dateBook}
\end{Highlighting}
\end{Shaded}

\begin{verbatim}
## [1] "2017-02-07"
\end{verbatim}

日期與字串的相關轉換操作可考慮使用簡單易懂的\texttt{lubridate}\citep{R-lubridate}
package,如果想要將\texttt{年/月/日}格式的文字轉換為日期物件,可使用\texttt{ymd()}函數(y表年year,m表月month,d表日day),如果想要將\texttt{月/日/年}格式的文字轉換為日期物件,則使用\texttt{mdy()}函數,以此類推。

\begin{Shaded}
\begin{Highlighting}[]
\KeywordTok{library}\NormalTok{(lubridate)}
\KeywordTok{ymd}\NormalTok{(}\StringTok{'2012/3/3'}\NormalTok{)}
\end{Highlighting}
\end{Shaded}

\begin{verbatim}
## [1] "2012-03-03"
\end{verbatim}

\begin{Shaded}
\begin{Highlighting}[]
\KeywordTok{mdy}\NormalTok{(}\StringTok{'3/3/2012'}\NormalTok{)}
\end{Highlighting}
\end{Shaded}

\begin{verbatim}
## [1] "2012-03-03"
\end{verbatim}

其他使用方式可參考
\href{http://blog.yhat.com/static/pdf/R_date_cheat_sheet.pdf}{The Yhat
Blog}。

\section{基本運算子}

\subsection{數學基本運算}

在R中,數學運算與其他程式語言相同

\begin{itemize}
\tightlist
\item
  加 \texttt{+}
\item
  減 \texttt{-}
\item
  乘 \texttt{*}
\item
  除 \texttt{/}
\item
  餘數 \texttt{\%\%}
\item
  次方 \texttt{\^{}}
\end{itemize}

\begin{Shaded}
\begin{Highlighting}[]
\NormalTok{num1<-}\DecValTok{1}
\NormalTok{num2<-}\DecValTok{100}
\NormalTok{num1+num2}
\end{Highlighting}
\end{Shaded}

\begin{verbatim}
## [1] 101
\end{verbatim}

\begin{Shaded}
\begin{Highlighting}[]
\NormalTok{num1-num2}
\end{Highlighting}
\end{Shaded}

\begin{verbatim}
## [1] -99
\end{verbatim}

\begin{Shaded}
\begin{Highlighting}[]
\NormalTok{num1*num2}
\end{Highlighting}
\end{Shaded}

\begin{verbatim}
## [1] 100
\end{verbatim}

\begin{Shaded}
\begin{Highlighting}[]
\NormalTok{num1/num2}
\end{Highlighting}
\end{Shaded}

\begin{verbatim}
## [1] 0.01
\end{verbatim}

\begin{Shaded}
\begin{Highlighting}[]
\DecValTok{100}\NormalTok\DecValTok{3} \NormalTok{##100除以3後所得餘數}
\end{Highlighting}
\end{Shaded}

\begin{verbatim}
## [1] 1
\end{verbatim}

\begin{Shaded}
\begin{Highlighting}[]
\DecValTok{2}\NormalTok{^}\DecValTok{3} \NormalTok{##2的3次方}
\end{Highlighting}
\end{Shaded}

\begin{verbatim}
## [1] 8
\end{verbatim}

\subsection{進階數學函數}

\begin{itemize}
\tightlist
\item
  四捨五入 \texttt{round()}
\item
  無條件捨去 \texttt{floor()}
\item
  無條件進位 \texttt{ceiling()}
\end{itemize}

\begin{Shaded}
\begin{Highlighting}[]
\NormalTok{num1<-}\FloatTok{1.568}
\NormalTok{num2<-}\FloatTok{2.121}
\KeywordTok{round}\NormalTok{(num1,}\DataTypeTok{digits =} \DecValTok{2}\NormalTok{) }\CommentTok{#四捨五入至小數點第二位}
\end{Highlighting}
\end{Shaded}

\begin{verbatim}
## [1] 1.57
\end{verbatim}

\begin{Shaded}
\begin{Highlighting}[]
\KeywordTok{round}\NormalTok{(num2,}\DataTypeTok{digits =} \DecValTok{1}\NormalTok{) }\CommentTok{#四捨五入至小數點第一位}
\end{Highlighting}
\end{Shaded}

\begin{verbatim}
## [1] 2.1
\end{verbatim}

\begin{Shaded}
\begin{Highlighting}[]
\KeywordTok{floor}\NormalTok{(num1) ##1.568}
\end{Highlighting}
\end{Shaded}

\begin{verbatim}
## [1] 1
\end{verbatim}

\begin{Shaded}
\begin{Highlighting}[]
\KeywordTok{ceiling}\NormalTok{(num2) ##2.121}
\end{Highlighting}
\end{Shaded}

\begin{verbatim}
## [1] 3
\end{verbatim}

\subsection{邏輯運算}

常用之邏輯判斷也可在R中直接使用

\begin{itemize}
\tightlist
\item
  大於 \texttt{\textgreater{}}
\item
  小於 \texttt{\textless{}}
\item
  等於
  \texttt{==},為了不與變數設定混淆,判斷兩變數是否相等,要用\textbf{雙等號}
\item
  大於等於 \texttt{\textgreater{}=}
\item
  小於等於 \texttt{\textless{}=}
\end{itemize}

\begin{Shaded}
\begin{Highlighting}[]
\NormalTok{num1<-}\DecValTok{1}
\NormalTok{num2<-}\DecValTok{100}
\NormalTok{num1>num2}
\end{Highlighting}
\end{Shaded}

\begin{verbatim}
## [1] FALSE
\end{verbatim}

\begin{Shaded}
\begin{Highlighting}[]
\NormalTok{num1<num2}
\end{Highlighting}
\end{Shaded}

\begin{verbatim}
## [1] TRUE
\end{verbatim}

文字字串也可比較大小

\begin{Shaded}
\begin{Highlighting}[]
\NormalTok{char1<-}\StringTok{"abcTest"} 
\NormalTok{char2<-}\StringTok{"defTest"}
\NormalTok{char1>char2}
\end{Highlighting}
\end{Shaded}

\begin{verbatim}
## [1] FALSE
\end{verbatim}

邏輯混合判斷,和JAVA等語言不同的是,在R中使用\textbf{單符號}即可表示且\texttt{\&}和或\texttt{\textbar{}}

\begin{itemize}
\tightlist
\item
  且 \texttt{\&}
\item
  或 \texttt{\textbar{}}
\end{itemize}

\begin{Shaded}
\begin{Highlighting}[]
\OtherTok{TRUE} \NormalTok{&}\StringTok{ }\OtherTok{TRUE}
\end{Highlighting}
\end{Shaded}

\begin{verbatim}
## [1] TRUE
\end{verbatim}

\begin{Shaded}
\begin{Highlighting}[]
\OtherTok{TRUE} \NormalTok{&}\StringTok{ }\OtherTok{FALSE}
\end{Highlighting}
\end{Shaded}

\begin{verbatim}
## [1] FALSE
\end{verbatim}

\begin{Shaded}
\begin{Highlighting}[]
\OtherTok{TRUE} \NormalTok{|}\StringTok{ }\OtherTok{TRUE}
\end{Highlighting}
\end{Shaded}

\begin{verbatim}
## [1] TRUE
\end{verbatim}

\begin{Shaded}
\begin{Highlighting}[]
\OtherTok{TRUE} \NormalTok{|}\StringTok{ }\OtherTok{FALSE}
\end{Highlighting}
\end{Shaded}

\begin{verbatim}
## [1] TRUE
\end{verbatim}

反向布林變數\texttt{!}

\begin{Shaded}
\begin{Highlighting}[]
\NormalTok{!}\OtherTok{TRUE}
\end{Highlighting}
\end{Shaded}

\begin{verbatim}
## [1] FALSE
\end{verbatim}

\begin{Shaded}
\begin{Highlighting}[]
\NormalTok{!}\OtherTok{FALSE}
\end{Highlighting}
\end{Shaded}

\begin{verbatim}
## [1] TRUE
\end{verbatim}

\section{錯誤訊息}

\begin{itemize}
\tightlist
\item
  Message:有可能的錯誤通知,程式會繼續執行
\item
  Warning:有錯誤,但是不會影響太多,程式會繼續執行
\item
  Error:有錯,而且無法繼續執行程式
\item
  Condition:可能會發生的情況
\end{itemize}

\begin{Shaded}
\begin{Highlighting}[]
\KeywordTok{log}\NormalTok{(-}\DecValTok{1}\NormalTok{)}
\end{Highlighting}
\end{Shaded}

\begin{verbatim}
## Warning in log(-1): NaNs produced
\end{verbatim}

\begin{verbatim}
## [1] NaN
\end{verbatim}

\begin{Shaded}
\begin{Highlighting}[]
\KeywordTok{mena}\NormalTok{(}\OtherTok{NA}\NormalTok{)}
\end{Highlighting}
\end{Shaded}

\begin{verbatim}
## Error in eval(expr, envir, enclos): could not find function "mena"
\end{verbatim}

錯誤訊息範例1:

\begin{verbatim}
# Error: could not find function "fetch_NBAPlayerStatistics"
# 找不到"fetch_NBAPlayerStatistics" function
\end{verbatim}

可能原因:沒安裝或沒讀入SportsAnalytics package

錯誤訊息範例2:

\begin{verbatim}
# Error in library(knitr): there is no package called 'knitr'
# 找不到"knitr" package
\end{verbatim}

可能原因:沒安裝knitr package

\section{Help}\label{help}

R語言與套件均有完整的文件與範例可以參考,在R的執行視窗中,輸入\texttt{?函數名稱}或\texttt{?套件名稱}即可看到函數或套件的使用說明

\begin{Shaded}
\begin{Highlighting}[]
\NormalTok{?ggplot2}
\NormalTok{?ymd}
\end{Highlighting}
\end{Shaded}

除此之外,\href{http://stackoverflow.com/}{Stack
Overflow}中也有許多問答,可直接在網站中搜尋關鍵字與錯誤訊息。

\chapter{R 資料結構}\label{RDataStructure}

\section{向量 vector}\label{-vector}

向量為一維資料的表現和儲存方式,用\texttt{c()}函數可定義向量,如:

\begin{Shaded}
\begin{Highlighting}[]
\NormalTok{vec<-}\KeywordTok{c}\NormalTok{(}\StringTok{'a'}\NormalTok{,}\StringTok{'b'}\NormalTok{,}\StringTok{'c'}\NormalTok{,}\StringTok{'d'}\NormalTok{,}\StringTok{'e'}\NormalTok{)}
\end{Highlighting}
\end{Shaded}

a\textasciitilde{}e為vec向量中的\textbf{元素(element)},各元素向中的順序固定,\texttt{a}為\texttt{vec}向量中的第\textbf{1}個元素,\texttt{b}則為第\textbf{2}個元素,以此類推,若要將\texttt{vec}向量的第\textbf{4}個元素取出,可使用

\begin{Shaded}
\begin{Highlighting}[]
\NormalTok{vec[}\DecValTok{4}\NormalTok{] ## 第4個元素}
\end{Highlighting}
\end{Shaded}

\begin{verbatim}
## [1] "d"
\end{verbatim}

也可同時取出多個元素

\begin{Shaded}
\begin{Highlighting}[]
\NormalTok{vec[}\KeywordTok{c}\NormalTok{(}\DecValTok{2}\NormalTok{,}\DecValTok{3}\NormalTok{)] ## 第2與第3個元素}
\end{Highlighting}
\end{Shaded}

\begin{verbatim}
## [1] "b" "c"
\end{verbatim}

此外,在同一向量中,所有元素之\textbf{資料型態必須相同},如上述\texttt{vec}向量,元素均為文字型態。

和變數指定類似,向量中的元素也可以使用\texttt{\textless{}-}重新指定

\begin{Shaded}
\begin{Highlighting}[]
\NormalTok{vec[}\DecValTok{3}\NormalTok{]}
\end{Highlighting}
\end{Shaded}

\begin{verbatim}
## [1] "c"
\end{verbatim}

\begin{Shaded}
\begin{Highlighting}[]
\NormalTok{vec[}\DecValTok{3}\NormalTok{]<-}\StringTok{'z'} \NormalTok{##第三個元素值設定為“z”}
\NormalTok{vec[}\DecValTok{3}\NormalTok{] }
\end{Highlighting}
\end{Shaded}

\begin{verbatim}
## [1] "z"
\end{verbatim}

\subsection{快速產生向量函數}

若要產生連續向量,如1\textasciitilde{}20,可使用\texttt{:}來串連首字與最後一字

\begin{Shaded}
\begin{Highlighting}[]
\DecValTok{1}\NormalTok{:}\DecValTok{20} \NormalTok{## c(1,2,...,19,20)}
\end{Highlighting}
\end{Shaded}

\begin{verbatim}
##  [1]  1  2  3  4  5  6  7  8  9 10 11 12 13 14 15 16 17 18 19 20
\end{verbatim}

或是使用\texttt{seq()}函數

\begin{Shaded}
\begin{Highlighting}[]
\KeywordTok{seq}\NormalTok{(}\DataTypeTok{from=}\DecValTok{1}\NormalTok{,}\DataTypeTok{to=}\DecValTok{20}\NormalTok{,}\DataTypeTok{by=}\DecValTok{1}\NormalTok{) ##1~20,中間相隔1}
\end{Highlighting}
\end{Shaded}

\begin{verbatim}
##  [1]  1  2  3  4  5  6  7  8  9 10 11 12 13 14 15 16 17 18 19 20
\end{verbatim}

\begin{Shaded}
\begin{Highlighting}[]
\KeywordTok{seq}\NormalTok{(}\DataTypeTok{from=}\DecValTok{1}\NormalTok{,}\DataTypeTok{to=}\DecValTok{50}\NormalTok{,}\DataTypeTok{by=}\DecValTok{2}\NormalTok{) ##1~50,中間相隔2}
\end{Highlighting}
\end{Shaded}

\begin{verbatim}
##  [1]  1  3  5  7  9 11 13 15 17 19 21 23 25 27 29 31 33 35 37 39 41 43 45 47 49
\end{verbatim}

\subsection{向量運算}

向量也可直接做加減乘除運算,如

\begin{Shaded}
\begin{Highlighting}[]
\NormalTok{numvec<-}\DecValTok{1}\NormalTok{:}\DecValTok{10} \NormalTok{## c(1,2,3,4,5,6,7,8,9,10)}
\NormalTok{numvec}\DecValTok{+3} \NormalTok{## 所有元素+3}
\end{Highlighting}
\end{Shaded}

\begin{verbatim}
##  [1]  4  5  6  7  8  9 10 11 12 13
\end{verbatim}

\begin{Shaded}
\begin{Highlighting}[]
\NormalTok{numvec*}\DecValTok{2} \NormalTok{## 所有元素*2}
\end{Highlighting}
\end{Shaded}

\begin{verbatim}
##  [1]  2  4  6  8 10 12 14 16 18 20
\end{verbatim}

向量和向量也可做運算,如

\begin{Shaded}
\begin{Highlighting}[]
\NormalTok{numvec1<-}\DecValTok{1}\NormalTok{:}\DecValTok{3} \NormalTok{## c(1,2,3)}
\NormalTok{numvec2<-}\DecValTok{4}\NormalTok{:}\DecValTok{6} \NormalTok{## c(4,5,6)}
\NormalTok{numvec1+numvec2}
\end{Highlighting}
\end{Shaded}

\begin{verbatim}
## [1] 5 7 9
\end{verbatim}

\begin{Shaded}
\begin{Highlighting}[]
\NormalTok{numvec1*numvec2}
\end{Highlighting}
\end{Shaded}

\begin{verbatim}
## [1]  4 10 18
\end{verbatim}

\section{因子 factor}\label{-factor}

因子是由向量轉換而成,多用於表示\textbf{類別}數據,如大學中有大學生、碩士班學生與博士班學生三種類別的學生,使用方法為\texttt{factor(資料向量,levels=類別次序)},\texttt{levels}參數可設定各類別的次序

\begin{Shaded}
\begin{Highlighting}[]
\KeywordTok{factor}\NormalTok{(}\KeywordTok{c}\NormalTok{(}\StringTok{"大學生"}\NormalTok{,}\StringTok{"碩士班學生"}\NormalTok{,}\StringTok{"博士班學生"}\NormalTok{),}
       \DataTypeTok{levels =} \KeywordTok{c}\NormalTok{(}\StringTok{"大學生"}\NormalTok{,}\StringTok{"碩士班學生"}\NormalTok{,}\StringTok{"博士班學生"}\NormalTok{))}
\end{Highlighting}
\end{Shaded}

\begin{verbatim}
## [1] 大學生     碩士班學生 博士班學生
## Levels: 大學生 碩士班學生 博士班學生
\end{verbatim}

因子變量一但決定其類別的種類與數目時,通常不會再作更動,也就是任何新增的元素都要是大學生、碩士班學生與博士班學生其中一種。

\section{列表 list}\label{-list}

由於向量和因子都只能儲存一種元素,使用上彈性較不足,在R語言中,有一彈性很大的資料型態\textbf{列表list},在列表中,元素可分屬不同資料類別,除了可包括\textbf{數值}與\textbf{文字}外,也可以包括資料集,如\textbf{向量}和\textbf{因子}等,更進階的使用,還可以包括矩陣與資料框。如要建立列表,可使用\texttt{list()}函數

\begin{Shaded}
\begin{Highlighting}[]
\NormalTok{listSample<-}\KeywordTok{list}\NormalTok{(}\DataTypeTok{Students=}\KeywordTok{c}\NormalTok{(}\StringTok{"Tom"}\NormalTok{,}\StringTok{"Kobe"}\NormalTok{,}\StringTok{"Emma"}\NormalTok{,}\StringTok{"Amy"}\NormalTok{),}\DataTypeTok{Year=}\DecValTok{2017}\NormalTok{,}
                 \DataTypeTok{Score=}\KeywordTok{c}\NormalTok{(}\DecValTok{60}\NormalTok{,}\DecValTok{50}\NormalTok{,}\DecValTok{80}\NormalTok{,}\DecValTok{40}\NormalTok{),}\DataTypeTok{School=}\StringTok{"CGU"}\NormalTok{)}
\NormalTok{listSample}
\end{Highlighting}
\end{Shaded}

\begin{verbatim}
## $Students
## [1] "Tom"  "Kobe" "Emma" "Amy" 
## 
## $Year
## [1] 2017
## 
## $Score
## [1] 60 50 80 40
## 
## $School
## [1] "CGU"
\end{verbatim}

\subsection{列表資料擷取}

列表可用\texttt{\$}符號做資料擷取

\begin{Shaded}
\begin{Highlighting}[]
\NormalTok{listSample$Students ##取得中表中的Students變量}
\end{Highlighting}
\end{Shaded}

\begin{verbatim}
## [1] "Tom"  "Kobe" "Emma" "Amy"
\end{verbatim}

也可和向量一樣,使用索引值來擷取資料,和向量不同的是,若要取得\textbf{值},要使用雙中括號\texttt{{[}{[}{]}{]}}

\begin{Shaded}
\begin{Highlighting}[]
\NormalTok{listSample[[}\DecValTok{1}\NormalTok{]] ##取得中表中第一個變量的值}
\end{Highlighting}
\end{Shaded}

\begin{verbatim}
## [1] "Tom"  "Kobe" "Emma" "Amy"
\end{verbatim}

如果只使用單中括號,回傳的資料型態會是列表list,並非列表中的值

\begin{Shaded}
\begin{Highlighting}[]
\NormalTok{listSample[}\DecValTok{1}\NormalTok{] ##取得中表中第一個變量(列表型態)}
\end{Highlighting}
\end{Shaded}

\begin{verbatim}
## $Students
## [1] "Tom"  "Kobe" "Emma" "Amy"
\end{verbatim}

\subsection{列表資料編輯設定}

列表資料也可和向量資料一樣,重新編輯設定

\begin{Shaded}
\begin{Highlighting}[]
\NormalTok{listSample[[}\DecValTok{1}\NormalTok{]] }
\end{Highlighting}
\end{Shaded}

\begin{verbatim}
## [1] "Tom"  "Kobe" "Emma" "Amy"
\end{verbatim}

\begin{Shaded}
\begin{Highlighting}[]
\NormalTok{listSample[[}\DecValTok{1}\NormalTok{]]<-}\KeywordTok{c}\NormalTok{(}\StringTok{"小明"}\NormalTok{,}\StringTok{"大雄"}\NormalTok{,}\StringTok{"胖虎"}\NormalTok{,}\StringTok{"小新"}\NormalTok{,}\StringTok{"大白"}\NormalTok{) ##將Students變量重新設定}
\NormalTok{listSample[[}\DecValTok{1}\NormalTok{]] }
\end{Highlighting}
\end{Shaded}

\begin{verbatim}
## [1] "小明" "大雄" "胖虎" "小新" "大白"
\end{verbatim}

除了編輯以外,列表資料也能用\texttt{\$}符號與\texttt{\textless{}-}變數設定符號新增

\begin{Shaded}
\begin{Highlighting}[]
\NormalTok{listSample$Gender<-}\KeywordTok{c}\NormalTok{(}\StringTok{"M"}\NormalTok{,}\StringTok{"F"}\NormalTok{,}\StringTok{"M"}\NormalTok{,}\StringTok{"F"}\NormalTok{,}\StringTok{"M"}\NormalTok{) ##新增Gender變量,並設定向量值}
\end{Highlighting}
\end{Shaded}

\section{矩陣 matrix}\label{-matrix}

\begin{Shaded}
\begin{Highlighting}[]
\NormalTok{a <-}\StringTok{ }\KeywordTok{matrix}\NormalTok{(}\KeywordTok{c}\NormalTok{(}\DecValTok{1}\NormalTok{:}\DecValTok{6}\NormalTok{), }\DataTypeTok{nrow=}\DecValTok{3}\NormalTok{, }\DataTypeTok{ncol=}\DecValTok{2}\NormalTok{) ##建立3x2的矩陣,分別填入1~6的值}
\NormalTok{a}
\end{Highlighting}
\end{Shaded}

\begin{verbatim}
##      [,1] [,2]
## [1,]    1    4
## [2,]    2    5
## [3,]    3    6
\end{verbatim}

\section{資料框 data.frame}\label{-data.frame}

資料框是非常常見的二維資料格式,由一系列的欄位(Column)和列(Row)所組成,常見的Excel試算表也是類似的資料表現形式,可使用\texttt{data.frame()}來創建新的資料框

\begin{Shaded}
\begin{Highlighting}[]
\NormalTok{StuDF <-}\StringTok{ }\KeywordTok{data.frame}\NormalTok{(}\DataTypeTok{StuID=}\KeywordTok{c}\NormalTok{(}\DecValTok{1}\NormalTok{,}\DecValTok{2}\NormalTok{,}\DecValTok{3}\NormalTok{,}\DecValTok{4}\NormalTok{,}\DecValTok{5}\NormalTok{), ##欄位名稱=欄位值}
                  \DataTypeTok{name=}\KeywordTok{c}\NormalTok{(}\StringTok{"小明"}\NormalTok{,}\StringTok{"大雄"}\NormalTok{,}\StringTok{"胖虎"}\NormalTok{,}\StringTok{"小新"}\NormalTok{,}\StringTok{"大白"}\NormalTok{),}
                  \DataTypeTok{score=}\KeywordTok{c}\NormalTok{(}\DecValTok{80}\NormalTok{,}\DecValTok{60}\NormalTok{,}\DecValTok{90}\NormalTok{,}\DecValTok{70}\NormalTok{,}\DecValTok{50}\NormalTok{))}
\NormalTok{StuDF }
\end{Highlighting}
\end{Shaded}

\begin{verbatim}
##   StuID name score
## 1     1 小明    80
## 2     2 大雄    60
## 3     3 胖虎    90
## 4     4 小新    70
## 5     5 大白    50
\end{verbatim}

如範例所示,每個欄位都有名稱(StuID, name,
score),若沒有設定欄位名稱,R會自動指派V1\textsubscript{Vn作為欄位名稱。在R中,每個欄位的資料型態必須相同,如StuID和score為數值型態,name為文字型態。每一列也有預設的列名,R自動依序指派1}n作為列名。
如需檢查欄位名稱與列名,可使用\texttt{colnames()}和\texttt{rownames()}

\begin{Shaded}
\begin{Highlighting}[]
\KeywordTok{colnames}\NormalTok{(StuDF) ##欄位名稱}
\end{Highlighting}
\end{Shaded}

\begin{verbatim}
## [1] "StuID" "name"  "score"
\end{verbatim}

\begin{Shaded}
\begin{Highlighting}[]
\KeywordTok{rownames}\NormalTok{(StuDF) ##列名}
\end{Highlighting}
\end{Shaded}

\begin{verbatim}
## [1] "1" "2" "3" "4" "5"
\end{verbatim}

如需檢查個欄位之資料型別,可使用\texttt{str()}函數

\begin{Shaded}
\begin{Highlighting}[]
\KeywordTok{str}\NormalTok{(StuDF) }
\end{Highlighting}
\end{Shaded}

\begin{verbatim}
## 'data.frame':    5 obs. of  3 variables:
##  $ StuID: num  1 2 3 4 5
##  $ name : Factor w/ 5 levels "大白","大雄",..: 4 2 5 3 1
##  $ score: num  80 60 90 70 50
\end{verbatim}

\section{資料表 data.table}\label{-data.table}

data.table是data.frame資料框型別的延伸,如要使用必須安裝data.table
\citep{R-data.table}
package,使用\texttt{data.table}讀取大型資料的速度比使用資料框快上數倍,進階處理語言也相當好用,在探索式資料分析章節Chapter
\ref{eda}會詳細介紹。其他詳細教學可見
\href{https://www.datacamp.com/community/tutorials/data-table-r-tutorial\#gs.vzMYa_k}{A
data.table R tutorial by
DataCamp},DataCamp也提供\href{https://www.datacamp.com/courses/data-table-data-manipulation-r-tutorial}{互動式教學課程},可自行參閱。

\section{資料屬性查詢函數}

資料屬性可透過下列函數查詢:

\begin{itemize}
\tightlist
\item
  名稱 \texttt{names()}
\item
  各維度名稱 \texttt{dimnames()}
\item
  長度 \texttt{length()}
\item
  各維度長度 \texttt{dim()}
\item
  資料型態 \texttt{class()}
\item
  各類資料計數 \texttt{table()}
\item
  總覽資料 \texttt{str()}
\end{itemize}

透過\texttt{names()}函數,可取得各種資料之名稱

\begin{Shaded}
\begin{Highlighting}[]
\KeywordTok{head}\NormalTok{(islands) ##R內建的資料}
\end{Highlighting}
\end{Shaded}

\begin{verbatim}
##       Africa   Antarctica         Asia    Australia Axel Heiberg       Baffin 
##        11506         5500        16988         2968           16          184
\end{verbatim}

\begin{Shaded}
\begin{Highlighting}[]
\KeywordTok{head}\NormalTok{(}\KeywordTok{names}\NormalTok{(islands)) ##顯示上述資料之資料名稱}
\end{Highlighting}
\end{Shaded}

\begin{verbatim}
## [1] "Africa"       "Antarctica"   "Asia"         "Australia"    "Axel Heiberg"
## [6] "Baffin"
\end{verbatim}

若為資料框,則會顯示行(欄位)名稱

\begin{Shaded}
\begin{Highlighting}[]
\KeywordTok{head}\NormalTok{(USArrests) ##R內建的資料}
\end{Highlighting}
\end{Shaded}

\begin{verbatim}
##            Murder Assault UrbanPop Rape
## Alabama      13.2     236       58 21.2
## Alaska       10.0     263       48 44.5
## Arizona       8.1     294       80 31.0
## Arkansas      8.8     190       50 19.5
## California    9.0     276       91 40.6
## Colorado      7.9     204       78 38.7
\end{verbatim}

\begin{Shaded}
\begin{Highlighting}[]
\KeywordTok{head}\NormalTok{(}\KeywordTok{names}\NormalTok{(USArrests)) ##顯示上述資料之資料名稱}
\end{Highlighting}
\end{Shaded}

\begin{verbatim}
## [1] "Murder"   "Assault"  "UrbanPop" "Rape"
\end{verbatim}

透過\texttt{dimnames()}函數可顯示資料框列與行的名稱,先顯示列,再顯示行

\begin{Shaded}
\begin{Highlighting}[]
\KeywordTok{dimnames}\NormalTok{(USArrests) }
\end{Highlighting}
\end{Shaded}

\begin{verbatim}
## [[1]]
##  [1] "Alabama"        "Alaska"         "Arizona"        "Arkansas"      
##  [5] "California"     "Colorado"       "Connecticut"    "Delaware"      
##  [9] "Florida"        "Georgia"        "Hawaii"         "Idaho"         
## [13] "Illinois"       "Indiana"        "Iowa"           "Kansas"        
## [17] "Kentucky"       "Louisiana"      "Maine"          "Maryland"      
## [21] "Massachusetts"  "Michigan"       "Minnesota"      "Mississippi"   
## [25] "Missouri"       "Montana"        "Nebraska"       "Nevada"        
## [29] "New Hampshire"  "New Jersey"     "New Mexico"     "New York"      
## [33] "North Carolina" "North Dakota"   "Ohio"           "Oklahoma"      
## [37] "Oregon"         "Pennsylvania"   "Rhode Island"   "South Carolina"
## [41] "South Dakota"   "Tennessee"      "Texas"          "Utah"          
## [45] "Vermont"        "Virginia"       "Washington"     "West Virginia" 
## [49] "Wisconsin"      "Wyoming"       
## 
## [[2]]
## [1] "Murder"   "Assault"  "UrbanPop" "Rape"
\end{verbatim}

透過\texttt{length()}函數可顯示資料長度,包括向量與資料框,若資料行態為資料框,則會顯示行(欄位)數

\begin{Shaded}
\begin{Highlighting}[]
\KeywordTok{length}\NormalTok{(islands) }
\end{Highlighting}
\end{Shaded}

\begin{verbatim}
## [1] 48
\end{verbatim}

\begin{Shaded}
\begin{Highlighting}[]
\KeywordTok{length}\NormalTok{(USArrests) }
\end{Highlighting}
\end{Shaded}

\begin{verbatim}
## [1] 4
\end{verbatim}

透過\texttt{dim()}函數可顯示資料框列與行的長度,與\texttt{dimnames()}相同,先顯示列,後顯示行

\begin{Shaded}
\begin{Highlighting}[]
\KeywordTok{dim}\NormalTok{(USArrests) }
\end{Highlighting}
\end{Shaded}

\begin{verbatim}
## [1] 50  4
\end{verbatim}

使用\texttt{class()}函數可知道變數類別

\begin{Shaded}
\begin{Highlighting}[]
\KeywordTok{class}\NormalTok{(}\DecValTok{1}\NormalTok{)}
\end{Highlighting}
\end{Shaded}

\begin{verbatim}
## [1] "numeric"
\end{verbatim}

\begin{Shaded}
\begin{Highlighting}[]
\KeywordTok{class}\NormalTok{(}\StringTok{"Test"}\NormalTok{)}
\end{Highlighting}
\end{Shaded}

\begin{verbatim}
## [1] "character"
\end{verbatim}

\begin{Shaded}
\begin{Highlighting}[]
\KeywordTok{class}\NormalTok{(}\KeywordTok{Sys.Date}\NormalTok{())}
\end{Highlighting}
\end{Shaded}

\begin{verbatim}
## [1] "Date"
\end{verbatim}

使用\texttt{table()}函數可知道向量中每個值出現幾次

\begin{Shaded}
\begin{Highlighting}[]
\NormalTok{iris$Species ##原始值}
\end{Highlighting}
\end{Shaded}

\begin{verbatim}
##   [1] setosa     setosa     setosa     setosa     setosa     setosa    
##   [7] setosa     setosa     setosa     setosa     setosa     setosa    
##  [13] setosa     setosa     setosa     setosa     setosa     setosa    
##  [19] setosa     setosa     setosa     setosa     setosa     setosa    
##  [25] setosa     setosa     setosa     setosa     setosa     setosa    
##  [31] setosa     setosa     setosa     setosa     setosa     setosa    
##  [37] setosa     setosa     setosa     setosa     setosa     setosa    
##  [43] setosa     setosa     setosa     setosa     setosa     setosa    
##  [49] setosa     setosa     versicolor versicolor versicolor versicolor
##  [55] versicolor versicolor versicolor versicolor versicolor versicolor
##  [61] versicolor versicolor versicolor versicolor versicolor versicolor
##  [67] versicolor versicolor versicolor versicolor versicolor versicolor
##  [73] versicolor versicolor versicolor versicolor versicolor versicolor
##  [79] versicolor versicolor versicolor versicolor versicolor versicolor
##  [85] versicolor versicolor versicolor versicolor versicolor versicolor
##  [91] versicolor versicolor versicolor versicolor versicolor versicolor
##  [97] versicolor versicolor versicolor versicolor virginica  virginica 
## [103] virginica  virginica  virginica  virginica  virginica  virginica 
## [109] virginica  virginica  virginica  virginica  virginica  virginica 
## [115] virginica  virginica  virginica  virginica  virginica  virginica 
## [121] virginica  virginica  virginica  virginica  virginica  virginica 
## [127] virginica  virginica  virginica  virginica  virginica  virginica 
## [133] virginica  virginica  virginica  virginica  virginica  virginica 
## [139] virginica  virginica  virginica  virginica  virginica  virginica 
## [145] virginica  virginica  virginica  virginica  virginica  virginica 
## Levels: setosa versicolor virginica
\end{verbatim}

\begin{Shaded}
\begin{Highlighting}[]
\KeywordTok{table}\NormalTok{(iris$Species) ##統計結果}
\end{Highlighting}
\end{Shaded}

\begin{verbatim}
## 
##     setosa versicolor  virginica 
##         50         50         50
\end{verbatim}

使用\texttt{str()}函數可總覽變數資訊

\begin{Shaded}
\begin{Highlighting}[]
\KeywordTok{str}\NormalTok{(iris)}
\end{Highlighting}
\end{Shaded}

\begin{verbatim}
## 'data.frame':    150 obs. of  5 variables:
##  $ Sepal.Length: num  5.1 4.9 4.7 4.6 5 5.4 4.6 5 4.4 4.9 ...
##  $ Sepal.Width : num  3.5 3 3.2 3.1 3.6 3.9 3.4 3.4 2.9 3.1 ...
##  $ Petal.Length: num  1.4 1.4 1.3 1.5 1.4 1.7 1.4 1.5 1.4 1.5 ...
##  $ Petal.Width : num  0.2 0.2 0.2 0.2 0.2 0.4 0.3 0.2 0.2 0.1 ...
##  $ Species     : Factor w/ 3 levels "setosa","versicolor",..: 1 1 1 1 1 1 1 1 1 1 ...
\end{verbatim}

\begin{Shaded}
\begin{Highlighting}[]
\KeywordTok{str}\NormalTok{(listSample)}
\end{Highlighting}
\end{Shaded}

\begin{verbatim}
## List of 5
##  $ Students: chr [1:5] "小明" "大雄" "胖虎" "小新" ...
##  $ Year    : num 2017
##  $ Score   : num [1:4] 60 50 80 40
##  $ School  : chr "CGU"
##  $ Gender  : chr [1:5] "M" "F" "M" "F" ...
\end{verbatim}

\chapter{控制流程}\label{controlstructure}

\section{條件判斷}

\subsection{if-else敘述}\label{if-else}

\textbf{if-else}敘述使用在邏輯判斷,若需要依條件改變需要執行的程式碼,就會使用\textbf{if-else},若\textbf{if}後所接邏輯判斷為\textbf{真(TRUE)},就會執行if下方之程式碼,若為\textbf{偽(FALSE)},則執行\textbf{else}下方之程式碼,若程式中沒有\textbf{else}片段,則不執行任何程式碼。

\begin{Shaded}
\begin{Highlighting}[]
\NormalTok{knitr::}\KeywordTok{include_graphics}\NormalTok{(}\StringTok{"figure/ifelse.png"}\NormalTok{)}
\end{Highlighting}
\end{Shaded}

\includegraphics{figure/ifelse.png}

\texttt{if}與\texttt{else}下方的程式碼必須要使用\texttt{\{\}}將程式碼包起來,若程式碼只有一行,可省略\texttt{\{\}},但為閱讀方便,建議不要省略\texttt{\{\}}。

舉例來說,若考試分數\textbf{大於等於60分},則印出\textbf{及格}字樣,小於60分則印出\textbf{不及格}字樣,程式範例如下:

\begin{Shaded}
\begin{Highlighting}[]
\NormalTok{score<-}\DecValTok{59}
\NormalTok{if(score>=}\DecValTok{60}\NormalTok{)\{}
  \KeywordTok{print}\NormalTok{(}\StringTok{"及格"}\NormalTok{)}
\NormalTok{\}else\{}
  \KeywordTok{print}\NormalTok{(}\StringTok{"不及格"}\NormalTok{)}
\NormalTok{\}}
\end{Highlighting}
\end{Shaded}

\begin{verbatim}
## [1] "不及格"
\end{verbatim}

\begin{Shaded}
\begin{Highlighting}[]
\NormalTok{score<-}\DecValTok{80}
\NormalTok{if(score>=}\DecValTok{60}\NormalTok{)\{}
  \KeywordTok{print}\NormalTok{(}\StringTok{"及格"}\NormalTok{)}
\NormalTok{\}else\{}
  \KeywordTok{print}\NormalTok{(}\StringTok{"不及格"}\NormalTok{)}
\NormalTok{\}}
\end{Highlighting}
\end{Shaded}

\begin{verbatim}
## [1] "及格"
\end{verbatim}

\subsection{if-else if-else}\label{if-else-if-else}

很多時候必須要使用多重邏輯判斷,若考試分數大於等於90分,印出\textbf{優良},介於60到90分間,印出\textbf{及格},小於60分則印出\textbf{不及格},此時就會用到多重邏輯,使用多重邏輯時,會在\texttt{if}和\texttt{else}間新增邏輯區段\textbf{else
if},程式範例如下:

\begin{Shaded}
\begin{Highlighting}[]
\NormalTok{score<-}\DecValTok{95}
\NormalTok{if(score>=}\DecValTok{90}\NormalTok{)\{}
  \KeywordTok{print}\NormalTok{(}\StringTok{"優秀"}\NormalTok{)}
\NormalTok{\}else if(score>=}\DecValTok{60}\NormalTok{)\{}
  \KeywordTok{print}\NormalTok{(}\StringTok{"及格"}\NormalTok{)}
\NormalTok{\}else\{}
  \KeywordTok{print}\NormalTok{(}\StringTok{"不及格"}\NormalTok{)}
\NormalTok{\}}
\end{Highlighting}
\end{Shaded}

\begin{verbatim}
## [1] "優秀"
\end{verbatim}

\texttt{if-else\ if-else}敘述是有順序的,若在\texttt{if}敘述判斷為真,就算後方\texttt{else\ if}判斷也為真,也只會執行\texttt{if}區段的程式碼,如上述範例,95分大於等於90分(if邏輯),也大於等於60分(else
if邏輯),但最後只印出\textbf{優秀}字樣。

\subsection{巢狀if}\label{if}

巢狀if是指在\texttt{if}區段程式碼內包含其他\texttt{if-else}判斷,舉例來說,若國文分數與英文分數皆大於等於60分,印出\textbf{全部及格},國文分數大於等於60分,英文小於60分,則印\textbf{國文及格,英文再加油},以此類推,程式範例如下:

\begin{Shaded}
\begin{Highlighting}[]
\NormalTok{CHscore<-}\DecValTok{95} \NormalTok{##國文成績}
\NormalTok{ENscore<-}\DecValTok{55} \NormalTok{##英文成績}
\NormalTok{if(CHscore>=}\DecValTok{60}\NormalTok{)\{}
  \NormalTok{if(ENscore>=}\DecValTok{60}\NormalTok{)\{}
    \KeywordTok{print}\NormalTok{(}\StringTok{"全部及格"}\NormalTok{)}
  \NormalTok{\}else\{}
    \KeywordTok{print}\NormalTok{(}\StringTok{"國文及格,英文再加油"}\NormalTok{)}
  \NormalTok{\}}
\NormalTok{\}else\{}
  \NormalTok{if(ENscore>=}\DecValTok{60}\NormalTok{)\{}
    \KeywordTok{print}\NormalTok{(}\StringTok{"英文及格,國文再加油"}\NormalTok{)}
  \NormalTok{\}else\{}
    \KeywordTok{print}\NormalTok{(}\StringTok{"全部不及格"}\NormalTok{)}
  \NormalTok{\}}
\NormalTok{\}}
\end{Highlighting}
\end{Shaded}

\begin{verbatim}
## [1] "國文及格,英文再加油"
\end{verbatim}

\subsection{ifelse()}\label{ifelse}

\texttt{ifelse()}函數可用最短的方式取代\texttt{if-else}敘述,使用方法為\texttt{ifelse(邏輯判斷,判斷為真要執行的程式碼,判斷為偽要執行的程式碼)},依上述範例,重寫程式碼如下:

\begin{Shaded}
\begin{Highlighting}[]
\NormalTok{score<-}\DecValTok{80}
\KeywordTok{ifelse}\NormalTok{(score>=}\DecValTok{60}\NormalTok{,}\StringTok{"及格"}\NormalTok{,}\StringTok{"不及格"}\NormalTok{)}
\end{Highlighting}
\end{Shaded}

\begin{verbatim}
## [1] "及格"
\end{verbatim}

值得注意的是,\texttt{ifelse()}可判斷向量,也就是可一次\textbf{判斷多個元素}

\begin{Shaded}
\begin{Highlighting}[]
\NormalTok{scoreVector<-}\KeywordTok{c}\NormalTok{(}\DecValTok{30}\NormalTok{,}\DecValTok{90}\NormalTok{,}\DecValTok{50}\NormalTok{,}\DecValTok{60}\NormalTok{,}\DecValTok{80}\NormalTok{)}
\KeywordTok{ifelse}\NormalTok{(scoreVector>=}\DecValTok{60}\NormalTok{,}\StringTok{"及格"}\NormalTok{, }\StringTok{"不及格"}\NormalTok{)}
\end{Highlighting}
\end{Shaded}

\begin{verbatim}
## [1] "不及格" "及格"   "不及格" "及格"   "及格"
\end{verbatim}

\section{迴圈}

\subsection{for}\label{for}

R語言的\texttt{for}迴圈寫法和其他語言不同,首先必須建立需要逐一執行的參數向量或序列,再使用\texttt{for}迴圈逐一執行,程式寫法為\texttt{for\ (單一變數\ in\ 參數向量)\{\ 程式碼\ \}},範例如下:

\begin{Shaded}
\begin{Highlighting}[]
\NormalTok{for (n in }\DecValTok{1}\NormalTok{:}\DecValTok{10}\NormalTok{)\{ }\CommentTok{#n為單一變數,1:10為需要逐一執行的參數向量}
  \KeywordTok{print}\NormalTok{(n)}
\NormalTok{\}}
\end{Highlighting}
\end{Shaded}

\begin{verbatim}
## [1] 1
## [1] 2
## [1] 3
## [1] 4
## [1] 5
## [1] 6
## [1] 7
## [1] 8
## [1] 9
## [1] 10
\end{verbatim}

\texttt{for}迴圈也可和\texttt{if-else}函數合併使用,如:

\begin{Shaded}
\begin{Highlighting}[]
\NormalTok{for (n in }\DecValTok{1}\NormalTok{:}\DecValTok{10}\NormalTok{)\{}
  \NormalTok{if(n%%}\DecValTok{2}\NormalTok{==}\DecValTok{0}\NormalTok{)\{ }\CommentTok{#偶數直接輸出數字}
    \KeywordTok{print}\NormalTok{(n)}
  \NormalTok{\}else\{}
    \KeywordTok{print}\NormalTok{(}\StringTok{"奇數"}\NormalTok{) }\CommentTok{#奇數則輸出"奇數"}
  \NormalTok{\}}
\NormalTok{\}}
\end{Highlighting}
\end{Shaded}

\begin{verbatim}
## [1] "奇數"
## [1] 2
## [1] "奇數"
## [1] 4
## [1] "奇數"
## [1] 6
## [1] "奇數"
## [1] 8
## [1] "奇數"
## [1] 10
\end{verbatim}

\subsection{while}\label{while}

\texttt{while}函數則是在每次執行迴圈時檢查while邏輯判斷是否為真,若邏輯判斷為真,就會執行區段程式碼,若邏輯判斷為偽,則會結束迴圈執行。

\begin{Shaded}
\begin{Highlighting}[]
\NormalTok{x<-}\DecValTok{0}
\NormalTok{while(x<=}\DecValTok{5}\NormalTok{)\{}
  \KeywordTok{print}\NormalTok{(x)}
  \NormalTok{x<-x}\DecValTok{+1}
\NormalTok{\}}
\end{Highlighting}
\end{Shaded}

\begin{verbatim}
## [1] 0
## [1] 1
## [1] 2
## [1] 3
## [1] 4
## [1] 5
\end{verbatim}

\subsection{break}\label{break}

若遇特殊情形想\textbf{結束}迴圈執行,可使用\texttt{break}指令

\begin{Shaded}
\begin{Highlighting}[]
\NormalTok{for(n in }\DecValTok{1}\NormalTok{:}\DecValTok{10}\NormalTok{)\{}
  \NormalTok{if(n==}\DecValTok{5}\NormalTok{)\{}
    \NormalTok{break ##一執行到5,跳出迴圈,不再執行之後的迴圈}
  \NormalTok{\}}
  \KeywordTok{print}\NormalTok{(n)}
\NormalTok{\}}
\end{Highlighting}
\end{Shaded}

\begin{verbatim}
## [1] 1
## [1] 2
## [1] 3
## [1] 4
\end{verbatim}

\subsection{next}\label{next}

若遇特殊情形想\textbf{跳過}迴圈執行,可使用\texttt{next}指令

\begin{Shaded}
\begin{Highlighting}[]
\NormalTok{for(n in }\DecValTok{1}\NormalTok{:}\DecValTok{10}\NormalTok{)\{}
  \NormalTok{if(n==}\DecValTok{5}\NormalTok{)\{}
    \NormalTok{next ##跳過5,直接執行下一個迴圈}
  \NormalTok{\}}
  \KeywordTok{print}\NormalTok{(n)}
\NormalTok{\}}
\end{Highlighting}
\end{Shaded}

\begin{verbatim}
## [1] 1
## [1] 2
## [1] 3
## [1] 4
## [1] 6
## [1] 7
## [1] 8
## [1] 9
## [1] 10
\end{verbatim}

\chapter{資料讀取與匯出}\label{io}

What is `Data'?

\url{http://en.wikipedia.org/wiki/Data}

\texttt{Data\ are\ values\ of\ qualitative\ or\ quantitative\ variables,\ belonging\ to\ a\ set\ of\ items.}

看完本章會學到:

\begin{itemize}
\tightlist
\item
  找資料、抓資料、總之就是弄到資料
\item
  資料前處理原則:tidy
\item
  用R做到以上工作
\end{itemize}

\texttt{Raw\ data} -\textgreater{} \textbf{Processing script}
-\textgreater{} \texttt{Tidy\ data} -\textgreater{} Data analysis
-\textgreater{} Data communication

\section{資料類型}

\subsection{Tidy Data}\label{tidy-data}

\begin{itemize}
\tightlist
\item
  一個欄位(Column)內只有一個數值,最好要有凡人看得懂的Column Name
\item
  不同的觀察值應該要在不同行(Raw)
\item
  一張表裡面,有所有分析需要的資料
\item
  如果一定要多張表,中間一定要有index可以把表串起來
\item
  One file, one table
\end{itemize}

\begin{Shaded}
\begin{Highlighting}[]
\NormalTok{if (!}\KeywordTok{require}\NormalTok{(}\StringTok{'SportsAnalytics'}\NormalTok{))\{}
    \KeywordTok{install.packages}\NormalTok{(}\StringTok{"SportsAnalytics"}\NormalTok{)}
    \KeywordTok{library}\NormalTok{(SportsAnalytics)}
\NormalTok{\}}
\NormalTok{NBA1415<-}\KeywordTok{fetch_NBAPlayerStatistics}\NormalTok{(}\StringTok{"14-15"}\NormalTok{)}
\end{Highlighting}
\end{Shaded}

\begin{Shaded}
\begin{Highlighting}[]
\KeywordTok{head}\NormalTok{(NBA1415)}
\end{Highlighting}
\end{Shaded}

\begin{verbatim}
##   League          Name Team Position GamesPlayed TotalMinutesPlayed
## 1    NBA    Quincy Acy  NYK       SF          68               1288
## 2    NBA  Jordan Adams  MEM       SG          30                249
## 3    NBA  Steven Adams  OKL        C          70               1776
## 4    NBA   Jeff Adrien  MIN       PF          17                215
## 5    NBA Arron Afflalo  POR       SG          78               2502
## 6    NBA Alexis Ajinca  NOR        C          68                956
##   FieldGoalsMade FieldGoalsAttempted ThreesMade ThreesAttempted FreeThrowsMade
## 1            152                 331         18              60             76
## 2             35                  86         10              25             14
## 3            217                 399          0               2            103
## 4             19                  44          0               0             22
## 5            375                 884        118             333            167
## 6            181                 328          0               0             81
##   FreeThrowsAttempted OffensiveRebounds TotalRebounds Assists Steals Turnovers
## 1                  97                79           301      68     27        60
## 2                  23                 9            28      16     16        14
## 3                 205               199           522      65     38        99
## 4                  38                23            77      15      4         9
## 5                 198                27           247     129     41       116
## 6                  99               104           315      47     21        69
##   Blocks PersonalFouls Disqualifications TotalPoints Technicals Ejections
## 1     22           147                 1         398          5         0
## 2      7            24                 0          94          0         0
## 3     85           222                 3         537          3         0
## 4      9            30                 0          60          0         0
## 5      7           167                 1        1035          0         0
## 6     51           151                 0         443          1         0
##   FlagrantFouls GamesStarted
## 1             0           22
## 2             0            0
## 3             0           67
## 4             0            0
## 5             0           72
## 6             0            8
\end{verbatim}

\subsection{真實世界裡的Raw Data}\label{raw-data}

\subsubsection{Html}\label{html}

\subsubsection{Facebook}\label{facebook}

\subsubsection{!?}\label{section}

\section{資料到底在哪裡?}

\begin{itemize}
\tightlist
\item
  硬碟裡
\item
  網路下載
\item
  Open Data

  \begin{itemize}
  \tightlist
  \item
    \url{http://data.taipei/}
  \item
    \url{http://data.tycg.gov.tw/}
  \item
    \url{http://data.moi.gov.tw/}
  \end{itemize}
\item
  網頁裡(爬蟲!)
\item
  任何地方
\end{itemize}

\subsection{Open Data}\label{open-data}

\section{讀取各類檔案}

\subsection{表格(.csv / Tab分隔 / Excel)}\label{.csv-tab-excel}

\texttt{read.table}, \texttt{read.csv}, 讀取表格資料

The \texttt{read.table}, \texttt{read.csv} 是最常見的,會用這些就好

有以下參數:

\begin{itemize}
\tightlist
\item
  \texttt{file}, 檔名
\item
  \texttt{header}, 是否有欄位名稱(表頭)(T/F)
\item
  \texttt{sep}, 分隔符號
\item
  \texttt{colClasses}, 每一個欄位的類別,用向量表示
\item
  \texttt{comment.char}, 把欄位包起來的符號
\item
  \texttt{skip}, 要跳過幾行?
\item
  \texttt{stringsAsFactors}, 要不要輸入成'Factor'(有序因子)
\end{itemize}

要安裝xlsx 套件

\begin{Shaded}
\begin{Highlighting}[]
\NormalTok{if (!}\KeywordTok{require}\NormalTok{(}\StringTok{'xlsx'}\NormalTok{))\{}
    \KeywordTok{install.packages}\NormalTok{(}\StringTok{"xlsx"}\NormalTok{)}
    \KeywordTok{library}\NormalTok{(xlsx)}
\NormalTok{\}}
\NormalTok{ExcelData <-}\StringTok{ }\KeywordTok{read.xlsx}\NormalTok{(}\StringTok{"data.xlsx"}\NormalTok{,}\DataTypeTok{sheetIndex=}\DecValTok{1}\NormalTok{,}\DataTypeTok{header=}\OtherTok{TRUE}\NormalTok{)}
\KeywordTok{head}\NormalTok{(ExcelData)}
\end{Highlighting}
\end{Shaded}

\section{read.csv使用範例}\label{read.csv}

完全不用給參數

\begin{Shaded}
\begin{Highlighting}[]
\NormalTok{data <-}\StringTok{ }\KeywordTok{read.csv}\NormalTok{(}\StringTok{'open.csv'}\NormalTok{)}
\NormalTok{data}
\end{Highlighting}
\end{Shaded}

\subsection{文字資料}

\texttt{readLines}, 逐行讀取文字資料

\subsection{R檔案格式}\label{r}

\texttt{load}, 讀取已經在R裡面的資料(Ex: iris)

\subsection{R程式}\label{r}

\texttt{source}, 讀R的Obejct or script, 執行, ASCII
(\texttt{dump}的相反)

\section{各類檔案匯出}

寫檔的函數跟讀檔很像:

\subsection{表格(.csv / Tab分隔)}\label{.csv-tab}

\texttt{write.table}

\subsection{文字資料}\label{-1}

\texttt{writeLines}

\subsection{R檔案格式}\label{r-1}

\texttt{save}

\subsection{R程式}\label{r-1}

\texttt{dump}

\section{讀檔的時候R會自動}\label{r}

\begin{itemize}
\tightlist
\item
  跳過\#開頭的任何行(Row)
\item
  判斷要讀幾行
\item
  判斷每個列(Column)的類別
\item
  把欄位包起來的符號
\end{itemize}

如果讀取時已指定\textbf{Column類別}以及\textbf{把欄位包起來的符號},會快很多

\begin{Shaded}
\begin{Highlighting}[]
\NormalTok{initial <-}\StringTok{ }\KeywordTok{read.csv}\NormalTok{(}\StringTok{"open.csv"}\NormalTok{, }\DataTypeTok{nrows =} \DecValTok{100}\NormalTok{)}
\NormalTok{classes <-}\StringTok{ }\KeywordTok{sapply}\NormalTok{(initial, class)}
\NormalTok{tabAll <-}\StringTok{ }\KeywordTok{read.csv}\NormalTok{(}\StringTok{"open.csv"}\NormalTok{, }\DataTypeTok{colClasses =} \NormalTok{classes)}
\end{Highlighting}
\end{Shaded}

\section{從網路上下載檔案 download.file}\label{-download.file}

使用RCurl Package

download.file(URL, destfile=儲存檔名, method=?)

method = ``curl''--\textgreater{}For \textbf{https}

\begin{Shaded}
\begin{Highlighting}[]
\NormalTok{if (!}\KeywordTok{require}\NormalTok{(}\StringTok{'RCurl'}\NormalTok{))\{}
    \KeywordTok{install.packages}\NormalTok{(}\StringTok{"RCurl"}\NormalTok{)}
    \KeywordTok{library}\NormalTok{(RCurl)}
\NormalTok{\}}
\KeywordTok{download.file}\NormalTok{(}\StringTok{"https://raw.githubusercontent.com/yijutseng/BigDataCGUIM/master/files/opendata10401.csv"}\NormalTok{, }
              \DataTypeTok{destfile =} \StringTok{"open.csv"}\NormalTok{, }\DataTypeTok{method =} \StringTok{"curl"}\NormalTok{)}
\end{Highlighting}
\end{Shaded}

\section{Open Data}\label{open-data-1}

\section{爬蟲}

\chapter{資料處理與清洗}\label{manipulation}

\section{資料型別轉換處理}

在資料型態章節Chapter \ref{DataType}中,曾介紹\textbf{數值
(numeric)}、\textbf{字串 (character)}、\textbf{布林變數
(logic)}以及\textbf{日期
(Date)}等資料型態,在此章節中將介紹如何檢查變數型別與各型別的轉換。

\subsection{資料型別檢查}

使用以下\texttt{is.}函數檢查資料型別,回傳布林變數,若為真,回傳TRUE

\begin{itemize}
\tightlist
\item
  是否為\textbf{數字} \texttt{is.numeric(變數名稱)}
\item
  是否為\textbf{文字} \texttt{is.character(變數名稱)}
\item
  是否為\textbf{布林變數} \texttt{is.logical(變數名稱)}
\end{itemize}

\begin{Shaded}
\begin{Highlighting}[]
\NormalTok{num<-}\DecValTok{100}
\NormalTok{cha<-}\StringTok{'200'}
\NormalTok{boo<-T}
\KeywordTok{is.numeric}\NormalTok{(num)}
\end{Highlighting}
\end{Shaded}

\begin{verbatim}
## [1] TRUE
\end{verbatim}

\begin{Shaded}
\begin{Highlighting}[]
\KeywordTok{is.numeric}\NormalTok{(cha)}
\end{Highlighting}
\end{Shaded}

\begin{verbatim}
## [1] FALSE
\end{verbatim}

\begin{Shaded}
\begin{Highlighting}[]
\KeywordTok{is.character}\NormalTok{(num)}
\end{Highlighting}
\end{Shaded}

\begin{verbatim}
## [1] FALSE
\end{verbatim}

\begin{Shaded}
\begin{Highlighting}[]
\KeywordTok{is.character}\NormalTok{(cha)}
\end{Highlighting}
\end{Shaded}

\begin{verbatim}
## [1] TRUE
\end{verbatim}

\begin{Shaded}
\begin{Highlighting}[]
\KeywordTok{is.logical}\NormalTok{(boo)}
\end{Highlighting}
\end{Shaded}

\begin{verbatim}
## [1] TRUE
\end{verbatim}

或使用\texttt{class(變數名稱)}函數,直接回傳資料型別

\begin{Shaded}
\begin{Highlighting}[]
\KeywordTok{class}\NormalTok{(num)}
\end{Highlighting}
\end{Shaded}

\begin{verbatim}
## [1] "numeric"
\end{verbatim}

\begin{Shaded}
\begin{Highlighting}[]
\KeywordTok{class}\NormalTok{(cha)}
\end{Highlighting}
\end{Shaded}

\begin{verbatim}
## [1] "character"
\end{verbatim}

\begin{Shaded}
\begin{Highlighting}[]
\KeywordTok{class}\NormalTok{(boo)}
\end{Highlighting}
\end{Shaded}

\begin{verbatim}
## [1] "logical"
\end{verbatim}

\begin{Shaded}
\begin{Highlighting}[]
\KeywordTok{class}\NormalTok{(}\KeywordTok{Sys.Date}\NormalTok{())}
\end{Highlighting}
\end{Shaded}

\begin{verbatim}
## [1] "Date"
\end{verbatim}

\subsection{資料型別轉換}

使用以下\texttt{as.}函數轉換型別

\begin{itemize}
\tightlist
\item
  轉換為\textbf{數字} \texttt{as.numeric(變數名稱)}
\item
  轉換為\textbf{文字} \texttt{as.character(變數名稱)}
\item
  轉換為\textbf{布林變數} \texttt{as.logical(變數名稱)}
\end{itemize}

\begin{Shaded}
\begin{Highlighting}[]
\KeywordTok{as.numeric}\NormalTok{(cha)}
\end{Highlighting}
\end{Shaded}

\begin{verbatim}
## [1] 200
\end{verbatim}

\begin{Shaded}
\begin{Highlighting}[]
\KeywordTok{as.numeric}\NormalTok{(boo)}
\end{Highlighting}
\end{Shaded}

\begin{verbatim}
## [1] 1
\end{verbatim}

\begin{Shaded}
\begin{Highlighting}[]
\KeywordTok{as.character}\NormalTok{(num)}
\end{Highlighting}
\end{Shaded}

\begin{verbatim}
## [1] "100"
\end{verbatim}

\begin{Shaded}
\begin{Highlighting}[]
\KeywordTok{as.character}\NormalTok{(boo)}
\end{Highlighting}
\end{Shaded}

\begin{verbatim}
## [1] "TRUE"
\end{verbatim}

若無法順利完成轉換,會回傳空值\texttt{NA},並出現警告訊息\texttt{Warning:\ NAs\ introduced\ by\ coercion,Warning:\ 強制變更過程中產生了\ NA}

\begin{Shaded}
\begin{Highlighting}[]
\KeywordTok{as.numeric}\NormalTok{(}\StringTok{"abc"}\NormalTok{)}
\end{Highlighting}
\end{Shaded}

\begin{verbatim}
## Warning: NAs introduced by coercion
\end{verbatim}

\begin{verbatim}
## [1] NA
\end{verbatim}

日期的轉換則建議使用\texttt{lubridate}\citep{R-lubridate}
package,如果想要將\texttt{年/月/日}格式的文字轉換為日期物件,可使用\texttt{ymd()}函數(y表年year,m表月month,d表日day),如果想要將\texttt{月/日/年}格式的文字轉換為日期物件,則使用\texttt{mdy()}函數,以此類推。

\begin{Shaded}
\begin{Highlighting}[]
\KeywordTok{library}\NormalTok{(lubridate)}
\KeywordTok{ymd}\NormalTok{(}\StringTok{'2012/3/3'}\NormalTok{)}
\end{Highlighting}
\end{Shaded}

\begin{verbatim}
## [1] "2012-03-03"
\end{verbatim}

\begin{Shaded}
\begin{Highlighting}[]
\KeywordTok{mdy}\NormalTok{(}\StringTok{'3/3/2012'}\NormalTok{)}
\end{Highlighting}
\end{Shaded}

\begin{verbatim}
## [1] "2012-03-03"
\end{verbatim}

\section{文字字串處理}

\subsection{基本處理}

\begin{itemize}
\tightlist
\item
  切割 \texttt{strsplit()}
\item
  子集 \texttt{substr()}
\item
  大小寫轉換 \texttt{toupper()} \texttt{tolower()}
\item
  兩文字連接 \texttt{paste()} \texttt{paste0()}
\item
  文字取代 \texttt{gsub()}
\item
  前後空白去除 \texttt{str\_trim()}
  需安裝\texttt{stringr}\citep{R-stringr} package
\end{itemize}

\begin{Shaded}
\begin{Highlighting}[]
\KeywordTok{strsplit} \NormalTok{(}\StringTok{"Hello World"}\NormalTok{,}\StringTok{" "}\NormalTok{)}
\end{Highlighting}
\end{Shaded}

\begin{verbatim}
## [[1]]
## [1] "Hello" "World"
\end{verbatim}

\begin{Shaded}
\begin{Highlighting}[]
\KeywordTok{toupper}\NormalTok{(}\StringTok{"Hello World"}\NormalTok{)}
\end{Highlighting}
\end{Shaded}

\begin{verbatim}
## [1] "HELLO WORLD"
\end{verbatim}

\begin{Shaded}
\begin{Highlighting}[]
\KeywordTok{tolower}\NormalTok{(}\StringTok{"Hello World"}\NormalTok{)}
\end{Highlighting}
\end{Shaded}

\begin{verbatim}
## [1] "hello world"
\end{verbatim}

\begin{Shaded}
\begin{Highlighting}[]
\KeywordTok{paste}\NormalTok{(}\StringTok{"Hello"}\NormalTok{, }\StringTok{"World"}\NormalTok{, }\DataTypeTok{sep=}\StringTok{''}\NormalTok{)}
\end{Highlighting}
\end{Shaded}

\begin{verbatim}
## [1] "HelloWorld"
\end{verbatim}

\begin{Shaded}
\begin{Highlighting}[]
\KeywordTok{substr}\NormalTok{(}\StringTok{"Hello World"}\NormalTok{, }\DataTypeTok{start=}\DecValTok{2}\NormalTok{,}\DataTypeTok{stop=}\DecValTok{4}\NormalTok{)}
\end{Highlighting}
\end{Shaded}

\begin{verbatim}
## [1] "ell"
\end{verbatim}

\begin{Shaded}
\begin{Highlighting}[]
\KeywordTok{gsub}\NormalTok{(}\StringTok{"o"}\NormalTok{,}\StringTok{"0"}\NormalTok{,}\StringTok{"Hello World"}\NormalTok{)}
\end{Highlighting}
\end{Shaded}

\begin{verbatim}
## [1] "Hell0 W0rld"
\end{verbatim}

\begin{Shaded}
\begin{Highlighting}[]
\KeywordTok{library}\NormalTok{(stringr)}
\KeywordTok{str_trim}\NormalTok{(}\StringTok{" Hello World "}\NormalTok{)}
\end{Highlighting}
\end{Shaded}

\begin{verbatim}
## [1] "Hello World"
\end{verbatim}

\subsection{搜尋字串}

搜尋字串函數通常使用在\textbf{比對文字向量},文字比對\textbf{有分大小寫},依照回傳值的型態不同,有兩種常用函數,\texttt{grep()}與\texttt{grepl()}:

\begin{itemize}
\tightlist
\item
  回傳符合條件之向量位置(index) \texttt{grep(搜尋條件,要搜尋的向量)}
\item
  回傳每個向量是否符合條件(TRUE or FALSE)
  \texttt{grepl(搜尋條件,要搜尋的向量)}
\end{itemize}

\begin{Shaded}
\begin{Highlighting}[]
\KeywordTok{grep}\NormalTok{(}\StringTok{"A"}\NormalTok{,}\KeywordTok{c}\NormalTok{(}\StringTok{"Alex"}\NormalTok{,}\StringTok{"Tom"}\NormalTok{,}\StringTok{"Amy"}\NormalTok{,}\StringTok{"Joy"}\NormalTok{,}\StringTok{"Emma"}\NormalTok{)) ##在姓名文字向量中尋找A,回傳包含"A"之元素位置}
\end{Highlighting}
\end{Shaded}

\begin{verbatim}
## [1] 1 3
\end{verbatim}

\begin{Shaded}
\begin{Highlighting}[]
\KeywordTok{grepl}\NormalTok{(}\StringTok{"A"}\NormalTok{,}\KeywordTok{c}\NormalTok{(}\StringTok{"Alex"}\NormalTok{,}\StringTok{"Tom"}\NormalTok{,}\StringTok{"Amy"}\NormalTok{,}\StringTok{"Joy"}\NormalTok{,}\StringTok{"Emma"}\NormalTok{)) ##在姓名文字向量中尋找A,回傳各元素是否包含"A"}
\end{Highlighting}
\end{Shaded}

\begin{verbatim}
## [1]  TRUE FALSE  TRUE FALSE FALSE
\end{verbatim}

\begin{Shaded}
\begin{Highlighting}[]
\KeywordTok{grepl}\NormalTok{(}\StringTok{"a"}\NormalTok{,}\KeywordTok{c}\NormalTok{(}\StringTok{"Alex"}\NormalTok{,}\StringTok{"Tom"}\NormalTok{,}\StringTok{"Amy"}\NormalTok{,}\StringTok{"Joy"}\NormalTok{,}\StringTok{"Emma"}\NormalTok{)) ##在姓名文字向量中尋找a,回傳各元素是否包含"a"}
\end{Highlighting}
\end{Shaded}

\begin{verbatim}
## [1] FALSE FALSE FALSE FALSE  TRUE
\end{verbatim}

\section{子集Subset}\label{subset}

\subsection{一維資料 (向量)}\label{-}

在向量章節\texttt{\{\#vector\}}有介紹使用\texttt{{[}{]}}取出單一或多個元素的方法

\begin{Shaded}
\begin{Highlighting}[]
\NormalTok{letters ##R語言內建資料之一}
\end{Highlighting}
\end{Shaded}

\begin{verbatim}
##  [1] "a" "b" "c" "d" "e" "f" "g" "h" "i" "j" "k" "l" "m" "n" "o" "p" "q" "r" "s"
## [20] "t" "u" "v" "w" "x" "y" "z"
\end{verbatim}

\begin{Shaded}
\begin{Highlighting}[]
\NormalTok{letters[}\DecValTok{1}\NormalTok{] ##取出letters向量的第一個元素}
\end{Highlighting}
\end{Shaded}

\begin{verbatim}
## [1] "a"
\end{verbatim}

\begin{Shaded}
\begin{Highlighting}[]
\NormalTok{letters[}\DecValTok{1}\NormalTok{:}\DecValTok{10}\NormalTok{] ##取出letters向量的前十個元素}
\end{Highlighting}
\end{Shaded}

\begin{verbatim}
##  [1] "a" "b" "c" "d" "e" "f" "g" "h" "i" "j"
\end{verbatim}

\begin{Shaded}
\begin{Highlighting}[]
\NormalTok{letters[}\KeywordTok{c}\NormalTok{(}\DecValTok{1}\NormalTok{,}\DecValTok{3}\NormalTok{,}\DecValTok{5}\NormalTok{)] ##取出letters向量的第1,3,5個元素}
\end{Highlighting}
\end{Shaded}

\begin{verbatim}
## [1] "a" "c" "e"
\end{verbatim}

\begin{Shaded}
\begin{Highlighting}[]
\NormalTok{letters[}\KeywordTok{c}\NormalTok{(-}\DecValTok{1}\NormalTok{,-}\DecValTok{3}\NormalTok{,-}\DecValTok{5}\NormalTok{)] ##取出letters向量除了第1,3,5個元素之外的所有元素}
\end{Highlighting}
\end{Shaded}

\begin{verbatim}
##  [1] "b" "d" "f" "g" "h" "i" "j" "k" "l" "m" "n" "o" "p" "q" "r" "s" "t" "u" "v"
## [20] "w" "x" "y" "z"
\end{verbatim}

若想要快速取得一向量的開頭與結尾元素,可使用\texttt{head()}和\texttt{tail()}函數

\begin{Shaded}
\begin{Highlighting}[]
\KeywordTok{head}\NormalTok{(letters,}\DecValTok{5}\NormalTok{) ##取出letters向量的前五個元素}
\end{Highlighting}
\end{Shaded}

\begin{verbatim}
## [1] "a" "b" "c" "d" "e"
\end{verbatim}

\begin{Shaded}
\begin{Highlighting}[]
\KeywordTok{tail}\NormalTok{(letters,}\DecValTok{3}\NormalTok{) ##取出letters向量的後三個元素}
\end{Highlighting}
\end{Shaded}

\begin{verbatim}
## [1] "x" "y" "z"
\end{verbatim}

\subsection{二維資料}

最常見的二維資料為data.frame資料框,二維資料可針對列(Row)和行(Column)做子集,子集選擇方式一樣是使用\texttt{{[}{]}},但因應二維資料的需求,以\texttt{,}分隔列與行的篩選條件,資料篩選原則為\textbf{前Row,後Column},\textbf{前列,後行},若不想篩選列,則在\texttt{,}前方保持\textbf{空白}即可。

篩選方式可輸入位置(index)、欄位名稱或輸入布林變數(TRUE/FALSE)

\begin{itemize}
\tightlist
\item
  輸入位置: \texttt{dataFrame{[}row\ index,column\ index{]}}
\item
  輸入布林變數: \texttt{dataFrame{[}c(T,F,T),c(T,F,T){]}}
\item
  輸入欄位名稱: \texttt{dataFrame{[}row\ name,column\ name{]}}
\end{itemize}

\begin{Shaded}
\begin{Highlighting}[]
\NormalTok{iris[}\DecValTok{1}\NormalTok{,}\DecValTok{2}\NormalTok{] ##第一列Row,第二行Column}
\end{Highlighting}
\end{Shaded}

\begin{verbatim}
## [1] 3.5
\end{verbatim}

\begin{Shaded}
\begin{Highlighting}[]
\NormalTok{iris[}\DecValTok{1}\NormalTok{:}\DecValTok{3}\NormalTok{,] ##第1~3列Row,所有的行Column}
\end{Highlighting}
\end{Shaded}

\begin{verbatim}
##   Sepal.Length Sepal.Width Petal.Length Petal.Width Species
## 1          5.1         3.5          1.4         0.2  setosa
## 2          4.9         3.0          1.4         0.2  setosa
## 3          4.7         3.2          1.3         0.2  setosa
\end{verbatim}

\begin{Shaded}
\begin{Highlighting}[]
\NormalTok{iris[,}\StringTok{"Species"}\NormalTok{] ##所有的列Row,名稱為Species的行Column}
\end{Highlighting}
\end{Shaded}

\begin{verbatim}
##   [1] setosa     setosa     setosa     setosa     setosa     setosa    
##   [7] setosa     setosa     setosa     setosa     setosa     setosa    
##  [13] setosa     setosa     setosa     setosa     setosa     setosa    
##  [19] setosa     setosa     setosa     setosa     setosa     setosa    
##  [25] setosa     setosa     setosa     setosa     setosa     setosa    
##  [31] setosa     setosa     setosa     setosa     setosa     setosa    
##  [37] setosa     setosa     setosa     setosa     setosa     setosa    
##  [43] setosa     setosa     setosa     setosa     setosa     setosa    
##  [49] setosa     setosa     versicolor versicolor versicolor versicolor
##  [55] versicolor versicolor versicolor versicolor versicolor versicolor
##  [61] versicolor versicolor versicolor versicolor versicolor versicolor
##  [67] versicolor versicolor versicolor versicolor versicolor versicolor
##  [73] versicolor versicolor versicolor versicolor versicolor versicolor
##  [79] versicolor versicolor versicolor versicolor versicolor versicolor
##  [85] versicolor versicolor versicolor versicolor versicolor versicolor
##  [91] versicolor versicolor versicolor versicolor versicolor versicolor
##  [97] versicolor versicolor versicolor versicolor virginica  virginica 
## [103] virginica  virginica  virginica  virginica  virginica  virginica 
## [109] virginica  virginica  virginica  virginica  virginica  virginica 
## [115] virginica  virginica  virginica  virginica  virginica  virginica 
## [121] virginica  virginica  virginica  virginica  virginica  virginica 
## [127] virginica  virginica  virginica  virginica  virginica  virginica 
## [133] virginica  virginica  virginica  virginica  virginica  virginica 
## [139] virginica  virginica  virginica  virginica  virginica  virginica 
## [145] virginica  virginica  virginica  virginica  virginica  virginica 
## Levels: setosa versicolor virginica
\end{verbatim}

\begin{Shaded}
\begin{Highlighting}[]
\NormalTok{iris[}\DecValTok{1}\NormalTok{:}\DecValTok{10}\NormalTok{,}\KeywordTok{c}\NormalTok{(T,F,T,F,T)] ##第1~10列Row,第1,3,5行Column (TRUE)}
\end{Highlighting}
\end{Shaded}

\begin{verbatim}
##    Sepal.Length Petal.Length Species
## 1           5.1          1.4  setosa
## 2           4.9          1.4  setosa
## 3           4.7          1.3  setosa
## 4           4.6          1.5  setosa
## 5           5.0          1.4  setosa
## 6           5.4          1.7  setosa
## 7           4.6          1.4  setosa
## 8           5.0          1.5  setosa
## 9           4.4          1.4  setosa
## 10          4.9          1.5  setosa
\end{verbatim}

也可使用\texttt{\$}符號做\textbf{Column的篩選}

\begin{Shaded}
\begin{Highlighting}[]
\NormalTok{iris$Species ##所有的列Row,名稱為Species的行Column}
\end{Highlighting}
\end{Shaded}

\begin{verbatim}
##   [1] setosa     setosa     setosa     setosa     setosa     setosa    
##   [7] setosa     setosa     setosa     setosa     setosa     setosa    
##  [13] setosa     setosa     setosa     setosa     setosa     setosa    
##  [19] setosa     setosa     setosa     setosa     setosa     setosa    
##  [25] setosa     setosa     setosa     setosa     setosa     setosa    
##  [31] setosa     setosa     setosa     setosa     setosa     setosa    
##  [37] setosa     setosa     setosa     setosa     setosa     setosa    
##  [43] setosa     setosa     setosa     setosa     setosa     setosa    
##  [49] setosa     setosa     versicolor versicolor versicolor versicolor
##  [55] versicolor versicolor versicolor versicolor versicolor versicolor
##  [61] versicolor versicolor versicolor versicolor versicolor versicolor
##  [67] versicolor versicolor versicolor versicolor versicolor versicolor
##  [73] versicolor versicolor versicolor versicolor versicolor versicolor
##  [79] versicolor versicolor versicolor versicolor versicolor versicolor
##  [85] versicolor versicolor versicolor versicolor versicolor versicolor
##  [91] versicolor versicolor versicolor versicolor versicolor versicolor
##  [97] versicolor versicolor versicolor versicolor virginica  virginica 
## [103] virginica  virginica  virginica  virginica  virginica  virginica 
## [109] virginica  virginica  virginica  virginica  virginica  virginica 
## [115] virginica  virginica  virginica  virginica  virginica  virginica 
## [121] virginica  virginica  virginica  virginica  virginica  virginica 
## [127] virginica  virginica  virginica  virginica  virginica  virginica 
## [133] virginica  virginica  virginica  virginica  virginica  virginica 
## [139] virginica  virginica  virginica  virginica  virginica  virginica 
## [145] virginica  virginica  virginica  virginica  virginica  virginica 
## Levels: setosa versicolor virginica
\end{verbatim}

\textbf{Row的篩選}可使用\texttt{subset()}函數,使用方法為\texttt{subset(資料表,篩選邏輯)}

\begin{Shaded}
\begin{Highlighting}[]
\KeywordTok{subset}\NormalTok{(iris,Species==}\StringTok{"virginica"}\NormalTok{) ##Species等於"virginica"的列Row,所有的行Column}
\end{Highlighting}
\end{Shaded}

\begin{verbatim}
##     Sepal.Length Sepal.Width Petal.Length Petal.Width   Species
## 101          6.3         3.3          6.0         2.5 virginica
## 102          5.8         2.7          5.1         1.9 virginica
## 103          7.1         3.0          5.9         2.1 virginica
## 104          6.3         2.9          5.6         1.8 virginica
## 105          6.5         3.0          5.8         2.2 virginica
## 106          7.6         3.0          6.6         2.1 virginica
## 107          4.9         2.5          4.5         1.7 virginica
## 108          7.3         2.9          6.3         1.8 virginica
## 109          6.7         2.5          5.8         1.8 virginica
## 110          7.2         3.6          6.1         2.5 virginica
## 111          6.5         3.2          5.1         2.0 virginica
## 112          6.4         2.7          5.3         1.9 virginica
## 113          6.8         3.0          5.5         2.1 virginica
## 114          5.7         2.5          5.0         2.0 virginica
## 115          5.8         2.8          5.1         2.4 virginica
## 116          6.4         3.2          5.3         2.3 virginica
## 117          6.5         3.0          5.5         1.8 virginica
## 118          7.7         3.8          6.7         2.2 virginica
## 119          7.7         2.6          6.9         2.3 virginica
## 120          6.0         2.2          5.0         1.5 virginica
## 121          6.9         3.2          5.7         2.3 virginica
## 122          5.6         2.8          4.9         2.0 virginica
## 123          7.7         2.8          6.7         2.0 virginica
## 124          6.3         2.7          4.9         1.8 virginica
## 125          6.7         3.3          5.7         2.1 virginica
## 126          7.2         3.2          6.0         1.8 virginica
## 127          6.2         2.8          4.8         1.8 virginica
## 128          6.1         3.0          4.9         1.8 virginica
## 129          6.4         2.8          5.6         2.1 virginica
## 130          7.2         3.0          5.8         1.6 virginica
## 131          7.4         2.8          6.1         1.9 virginica
## 132          7.9         3.8          6.4         2.0 virginica
## 133          6.4         2.8          5.6         2.2 virginica
## 134          6.3         2.8          5.1         1.5 virginica
## 135          6.1         2.6          5.6         1.4 virginica
## 136          7.7         3.0          6.1         2.3 virginica
## 137          6.3         3.4          5.6         2.4 virginica
## 138          6.4         3.1          5.5         1.8 virginica
## 139          6.0         3.0          4.8         1.8 virginica
## 140          6.9         3.1          5.4         2.1 virginica
## 141          6.7         3.1          5.6         2.4 virginica
## 142          6.9         3.1          5.1         2.3 virginica
## 143          5.8         2.7          5.1         1.9 virginica
## 144          6.8         3.2          5.9         2.3 virginica
## 145          6.7         3.3          5.7         2.5 virginica
## 146          6.7         3.0          5.2         2.3 virginica
## 147          6.3         2.5          5.0         1.9 virginica
## 148          6.5         3.0          5.2         2.0 virginica
## 149          6.2         3.4          5.4         2.3 virginica
## 150          5.9         3.0          5.1         1.8 virginica
\end{verbatim}

\textbf{Row的篩選}也可搭配字串搜尋函數\texttt{grepl()}

\begin{Shaded}
\begin{Highlighting}[]
\NormalTok{knitr::}\KeywordTok{kable}\NormalTok{(iris[}\KeywordTok{grepl}\NormalTok{(}\StringTok{"color"}\NormalTok{,iris$Species),]) ##Species包含"color"的列,所有的行}
\end{Highlighting}
\end{Shaded}

\begin{tabular}{l|r|r|r|r|l}
\hline
  & Sepal.Length & Sepal.Width & Petal.Length & Petal.Width & Species\\
\hline
51 & 7.0 & 3.2 & 4.7 & 1.4 & versicolor\\
\hline
52 & 6.4 & 3.2 & 4.5 & 1.5 & versicolor\\
\hline
53 & 6.9 & 3.1 & 4.9 & 1.5 & versicolor\\
\hline
54 & 5.5 & 2.3 & 4.0 & 1.3 & versicolor\\
\hline
55 & 6.5 & 2.8 & 4.6 & 1.5 & versicolor\\
\hline
56 & 5.7 & 2.8 & 4.5 & 1.3 & versicolor\\
\hline
57 & 6.3 & 3.3 & 4.7 & 1.6 & versicolor\\
\hline
58 & 4.9 & 2.4 & 3.3 & 1.0 & versicolor\\
\hline
59 & 6.6 & 2.9 & 4.6 & 1.3 & versicolor\\
\hline
60 & 5.2 & 2.7 & 3.9 & 1.4 & versicolor\\
\hline
61 & 5.0 & 2.0 & 3.5 & 1.0 & versicolor\\
\hline
62 & 5.9 & 3.0 & 4.2 & 1.5 & versicolor\\
\hline
63 & 6.0 & 2.2 & 4.0 & 1.0 & versicolor\\
\hline
64 & 6.1 & 2.9 & 4.7 & 1.4 & versicolor\\
\hline
65 & 5.6 & 2.9 & 3.6 & 1.3 & versicolor\\
\hline
66 & 6.7 & 3.1 & 4.4 & 1.4 & versicolor\\
\hline
67 & 5.6 & 3.0 & 4.5 & 1.5 & versicolor\\
\hline
68 & 5.8 & 2.7 & 4.1 & 1.0 & versicolor\\
\hline
69 & 6.2 & 2.2 & 4.5 & 1.5 & versicolor\\
\hline
70 & 5.6 & 2.5 & 3.9 & 1.1 & versicolor\\
\hline
71 & 5.9 & 3.2 & 4.8 & 1.8 & versicolor\\
\hline
72 & 6.1 & 2.8 & 4.0 & 1.3 & versicolor\\
\hline
73 & 6.3 & 2.5 & 4.9 & 1.5 & versicolor\\
\hline
74 & 6.1 & 2.8 & 4.7 & 1.2 & versicolor\\
\hline
75 & 6.4 & 2.9 & 4.3 & 1.3 & versicolor\\
\hline
76 & 6.6 & 3.0 & 4.4 & 1.4 & versicolor\\
\hline
77 & 6.8 & 2.8 & 4.8 & 1.4 & versicolor\\
\hline
78 & 6.7 & 3.0 & 5.0 & 1.7 & versicolor\\
\hline
79 & 6.0 & 2.9 & 4.5 & 1.5 & versicolor\\
\hline
80 & 5.7 & 2.6 & 3.5 & 1.0 & versicolor\\
\hline
81 & 5.5 & 2.4 & 3.8 & 1.1 & versicolor\\
\hline
82 & 5.5 & 2.4 & 3.7 & 1.0 & versicolor\\
\hline
83 & 5.8 & 2.7 & 3.9 & 1.2 & versicolor\\
\hline
84 & 6.0 & 2.7 & 5.1 & 1.6 & versicolor\\
\hline
85 & 5.4 & 3.0 & 4.5 & 1.5 & versicolor\\
\hline
86 & 6.0 & 3.4 & 4.5 & 1.6 & versicolor\\
\hline
87 & 6.7 & 3.1 & 4.7 & 1.5 & versicolor\\
\hline
88 & 6.3 & 2.3 & 4.4 & 1.3 & versicolor\\
\hline
89 & 5.6 & 3.0 & 4.1 & 1.3 & versicolor\\
\hline
90 & 5.5 & 2.5 & 4.0 & 1.3 & versicolor\\
\hline
91 & 5.5 & 2.6 & 4.4 & 1.2 & versicolor\\
\hline
92 & 6.1 & 3.0 & 4.6 & 1.4 & versicolor\\
\hline
93 & 5.8 & 2.6 & 4.0 & 1.2 & versicolor\\
\hline
94 & 5.0 & 2.3 & 3.3 & 1.0 & versicolor\\
\hline
95 & 5.6 & 2.7 & 4.2 & 1.3 & versicolor\\
\hline
96 & 5.7 & 3.0 & 4.2 & 1.2 & versicolor\\
\hline
97 & 5.7 & 2.9 & 4.2 & 1.3 & versicolor\\
\hline
98 & 6.2 & 2.9 & 4.3 & 1.3 & versicolor\\
\hline
99 & 5.1 & 2.5 & 3.0 & 1.1 & versicolor\\
\hline
100 & 5.7 & 2.8 & 4.1 & 1.3 & versicolor\\
\hline
\end{tabular}

若想要快速取得資料框的前幾列(Raw)或後幾列,也可使用\texttt{head()}和\texttt{tail()}函數

\begin{Shaded}
\begin{Highlighting}[]
\KeywordTok{head}\NormalTok{(iris,}\DecValTok{5}\NormalTok{) ##取出iris資料框的前五列}
\end{Highlighting}
\end{Shaded}

\begin{verbatim}
##   Sepal.Length Sepal.Width Petal.Length Petal.Width Species
## 1          5.1         3.5          1.4         0.2  setosa
## 2          4.9         3.0          1.4         0.2  setosa
## 3          4.7         3.2          1.3         0.2  setosa
## 4          4.6         3.1          1.5         0.2  setosa
## 5          5.0         3.6          1.4         0.2  setosa
\end{verbatim}

\begin{Shaded}
\begin{Highlighting}[]
\KeywordTok{tail}\NormalTok{(iris,}\DecValTok{3}\NormalTok{) ##取出iris資料框的後三列}
\end{Highlighting}
\end{Shaded}

\begin{verbatim}
##     Sepal.Length Sepal.Width Petal.Length Petal.Width   Species
## 148          6.5         3.0          5.2         2.0 virginica
## 149          6.2         3.4          5.4         2.3 virginica
## 150          5.9         3.0          5.1         1.8 virginica
\end{verbatim}

\section{排序}

\subsection{sort 向量排序}\label{sort-}

\texttt{sort()}函數可直接對向量做\textbf{由小到大}的排序

\begin{Shaded}
\begin{Highlighting}[]
\KeywordTok{head}\NormalTok{(islands) ##排序前的前六筆資料}
\end{Highlighting}
\end{Shaded}

\begin{verbatim}
##       Africa   Antarctica         Asia    Australia Axel Heiberg       Baffin 
##        11506         5500        16988         2968           16          184
\end{verbatim}

\begin{Shaded}
\begin{Highlighting}[]
\KeywordTok{head}\NormalTok{(}\KeywordTok{sort}\NormalTok{(islands)) ##由小到大排序後的前六筆資料}
\end{Highlighting}
\end{Shaded}

\begin{verbatim}
##       Vancouver          Hainan Prince of Wales           Timor          Kyushu 
##              12              13              13              13              14 
##          Taiwan 
##              14
\end{verbatim}

如需\textbf{由大到小}排序,可將\texttt{decreasing}參數設為TRUE

\begin{Shaded}
\begin{Highlighting}[]
\KeywordTok{head}\NormalTok{(}\KeywordTok{sort}\NormalTok{(islands,}\DataTypeTok{decreasing =} \NormalTok{T)) ##由大到小排序後的前六筆資料}
\end{Highlighting}
\end{Shaded}

\begin{verbatim}
##          Asia        Africa North America South America    Antarctica 
##         16988         11506          9390          6795          5500 
##        Europe 
##          3745
\end{verbatim}

\subsection{order}\label{order}

如需對資料框做排序,可使用\texttt{order()}函數,\texttt{order()}函數可回傳\textbf{由小到大}之\textbf{元素位置},以\texttt{iris\$Sepal.Length}為例,回傳的第一個位置為\texttt{14},表示\texttt{iris\$Sepal.Length}中,數值最小的元素為第14個元素。

\begin{Shaded}
\begin{Highlighting}[]
\KeywordTok{order}\NormalTok{(iris$Sepal.Length)}
\end{Highlighting}
\end{Shaded}

\begin{verbatim}
##   [1]  14   9  39  43  42   4   7  23  48   3  30  12  13  25  31  46   2  10
##  [19]  35  38  58 107   5   8  26  27  36  41  44  50  61  94   1  18  20  22
##  [37]  24  40  45  47  99  28  29  33  60  49   6  11  17  21  32  85  34  37
##  [55]  54  81  82  90  91  65  67  70  89  95 122  16  19  56  80  96  97 100
##  [73] 114  15  68  83  93 102 115 143  62  71 150  63  79  84  86 120 139  64
##  [91]  72  74  92 128 135  69  98 127 149  57  73  88 101 104 124 134 137 147
## [109]  52  75 112 116 129 133 138  55 105 111 117 148  59  76  66  78  87 109
## [127] 125 141 145 146  77 113 144  53 121 140 142  51 103 110 126 130 108 131
## [145] 106 118 119 123 136 132
\end{verbatim}

\begin{Shaded}
\begin{Highlighting}[]
\NormalTok{iris$Sepal.Length[}\DecValTok{14}\NormalTok{]}
\end{Highlighting}
\end{Shaded}

\begin{verbatim}
## [1] 4.3
\end{verbatim}

若將\texttt{decreasing}參數設定為TRUE,則會回傳\textbf{由大到小}的元素位置,以\texttt{iris\$Sepal.Length}為例,回傳的第一個位置為\texttt{132},表示\texttt{iris\$Sepal.Length}中,數值最大的元素為第132個元素。

\begin{Shaded}
\begin{Highlighting}[]
\KeywordTok{order}\NormalTok{(iris$Sepal.Length,}\DataTypeTok{decreasing =} \NormalTok{T)}
\end{Highlighting}
\end{Shaded}

\begin{verbatim}
##   [1] 132 118 119 123 136 106 131 108 110 126 130 103  51  53 121 140 142  77
##  [19] 113 144  66  78  87 109 125 141 145 146  59  76  55 105 111 117 148  52
##  [37]  75 112 116 129 133 138  57  73  88 101 104 124 134 137 147  69  98 127
##  [55] 149  64  72  74  92 128 135  63  79  84  86 120 139  62  71 150  15  68
##  [73]  83  93 102 115 143  16  19  56  80  96  97 100 114  65  67  70  89  95
##  [91] 122  34  37  54  81  82  90  91   6  11  17  21  32  85  49  28  29  33
## [109]  60   1  18  20  22  24  40  45  47  99   5   8  26  27  36  41  44  50
## [127]  61  94   2  10  35  38  58 107  12  13  25  31  46   3  30   4   7  23
## [145]  48  42   9  39  43  14
\end{verbatim}

\begin{Shaded}
\begin{Highlighting}[]
\NormalTok{iris$Sepal.Length[}\DecValTok{132}\NormalTok{]}
\end{Highlighting}
\end{Shaded}

\begin{verbatim}
## [1] 7.9
\end{verbatim}

依照order回傳的元素位置,重新排序iris資料框

\begin{Shaded}
\begin{Highlighting}[]
\KeywordTok{head}\NormalTok{(iris) ##排序前的前六筆資料}
\end{Highlighting}
\end{Shaded}

\begin{verbatim}
##   Sepal.Length Sepal.Width Petal.Length Petal.Width Species
## 1          5.1         3.5          1.4         0.2  setosa
## 2          4.9         3.0          1.4         0.2  setosa
## 3          4.7         3.2          1.3         0.2  setosa
## 4          4.6         3.1          1.5         0.2  setosa
## 5          5.0         3.6          1.4         0.2  setosa
## 6          5.4         3.9          1.7         0.4  setosa
\end{verbatim}

\begin{Shaded}
\begin{Highlighting}[]
\KeywordTok{head}\NormalTok{(iris[}\KeywordTok{order}\NormalTok{(iris$Sepal.Length),]) ##依照Sepal.Length欄位數值大小排序後的前六筆資料}
\end{Highlighting}
\end{Shaded}

\begin{verbatim}
##    Sepal.Length Sepal.Width Petal.Length Petal.Width Species
## 14          4.3         3.0          1.1         0.1  setosa
## 9           4.4         2.9          1.4         0.2  setosa
## 39          4.4         3.0          1.3         0.2  setosa
## 43          4.4         3.2          1.3         0.2  setosa
## 42          4.5         2.3          1.3         0.3  setosa
## 4           4.6         3.1          1.5         0.2  setosa
\end{verbatim}

\begin{Shaded}
\begin{Highlighting}[]
\KeywordTok{head}\NormalTok{(iris[}\KeywordTok{order}\NormalTok{(iris$Sepal.Length,}\DataTypeTok{decreasing =} \NormalTok{T),]) ##改為由大到小排序的前六筆資料}
\end{Highlighting}
\end{Shaded}

\begin{verbatim}
##     Sepal.Length Sepal.Width Petal.Length Petal.Width   Species
## 132          7.9         3.8          6.4         2.0 virginica
## 118          7.7         3.8          6.7         2.2 virginica
## 119          7.7         2.6          6.9         2.3 virginica
## 123          7.7         2.8          6.7         2.0 virginica
## 136          7.7         3.0          6.1         2.3 virginica
## 106          7.6         3.0          6.6         2.1 virginica
\end{verbatim}

\section{資料組合}

有時需要在資料框新增一列,或新增一行,可以利用資料組合函數完成

\begin{itemize}
\tightlist
\item
  Row 列的組合 \texttt{rbind()}
\item
  Column 行的組合 \texttt{cbind()}
\end{itemize}

\texttt{rbind()}和\texttt{cbind()}的參數可以是向量,也可以是資料框,使用向量做資料整合範例:

\begin{Shaded}
\begin{Highlighting}[]
\KeywordTok{rbind}\NormalTok{(}\KeywordTok{c}\NormalTok{(}\DecValTok{1}\NormalTok{,}\DecValTok{2}\NormalTok{,}\DecValTok{3}\NormalTok{), }\CommentTok{#第一列}
      \KeywordTok{c}\NormalTok{(}\DecValTok{4}\NormalTok{,}\DecValTok{5}\NormalTok{,}\DecValTok{6}\NormalTok{)  }\CommentTok{#第二列}
      \NormalTok{) }
\end{Highlighting}
\end{Shaded}

\begin{verbatim}
##      [,1] [,2] [,3]
## [1,]    1    2    3
## [2,]    4    5    6
\end{verbatim}

使用資料框與向量做資料整合範例:

\begin{Shaded}
\begin{Highlighting}[]
\NormalTok{irisAdd<-}\KeywordTok{rbind}\NormalTok{(iris, }\CommentTok{#資料框}
      \KeywordTok{c}\NormalTok{(}\DecValTok{1}\NormalTok{,}\DecValTok{1}\NormalTok{,}\DecValTok{1}\NormalTok{,}\DecValTok{1}\NormalTok{,}\StringTok{"versicolor"}\NormalTok{)  }\CommentTok{#新增一列}
      \NormalTok{) }
\KeywordTok{tail}\NormalTok{(irisAdd)}
\end{Highlighting}
\end{Shaded}

\begin{verbatim}
##     Sepal.Length Sepal.Width Petal.Length Petal.Width    Species
## 146          6.7           3          5.2         2.3  virginica
## 147          6.3         2.5            5         1.9  virginica
## 148          6.5           3          5.2           2  virginica
## 149          6.2         3.4          5.4         2.3  virginica
## 150          5.9           3          5.1         1.8  virginica
## 151            1           1            1           1 versicolor
\end{verbatim}

使用向量做資料整合範例:

\begin{Shaded}
\begin{Highlighting}[]
\KeywordTok{cbind}\NormalTok{(}\KeywordTok{c}\NormalTok{(}\DecValTok{1}\NormalTok{,}\DecValTok{2}\NormalTok{,}\DecValTok{3}\NormalTok{), }\CommentTok{#第一行}
      \KeywordTok{c}\NormalTok{(}\DecValTok{4}\NormalTok{,}\DecValTok{5}\NormalTok{,}\DecValTok{6}\NormalTok{)  }\CommentTok{#第二行}
      \NormalTok{) }
\end{Highlighting}
\end{Shaded}

\begin{verbatim}
##      [,1] [,2]
## [1,]    1    4
## [2,]    2    5
## [3,]    3    6
\end{verbatim}

使用資料框與向量做資料整合範例:

\begin{Shaded}
\begin{Highlighting}[]
\NormalTok{irisAdd<-}\KeywordTok{cbind}\NormalTok{(iris, }\CommentTok{#資料框}
      \KeywordTok{rep}\NormalTok{(}\StringTok{"Add"}\NormalTok{,}\KeywordTok{nrow}\NormalTok{(iris))  }\CommentTok{#新增一行}
      \NormalTok{) }
\KeywordTok{tail}\NormalTok{(irisAdd)}
\end{Highlighting}
\end{Shaded}

\begin{verbatim}
##     Sepal.Length Sepal.Width Petal.Length Petal.Width   Species
## 145          6.7         3.3          5.7         2.5 virginica
## 146          6.7         3.0          5.2         2.3 virginica
## 147          6.3         2.5          5.0         1.9 virginica
## 148          6.5         3.0          5.2         2.0 virginica
## 149          6.2         3.4          5.4         2.3 virginica
## 150          5.9         3.0          5.1         1.8 virginica
##     rep("Add", nrow(iris))
## 145                    Add
## 146                    Add
## 147                    Add
## 148                    Add
## 149                    Add
## 150                    Add
\end{verbatim}

\section{長表與寬表}

在資料處理的過程中,常因各種需求,需要執行長寬表互換的動作,在R中有很好用的套件reshape2\citep{R-reshape2}
package,提供完整的轉換功能,最常使用的是

\begin{itemize}
\tightlist
\item
  寬表轉長表 \texttt{melt(資料框/寬表,id.vars=需要保留的欄位)}
\item
  長表轉寬表
  \texttt{dcast(資料框/長表,寬表分列依據\textasciitilde{}分欄位依據)}
\end{itemize}

原來的\texttt{airquality}資料框中,有Ozone, Solar.R, Wind, Temp, Month,
Day等六個欄位
(Column),屬於寬表,以下範例將保留Month和Day兩個欄位,並將其他欄位的名稱整合至variable欄位,數值整合至value欄位,寬表轉長表範例如下:

\begin{Shaded}
\begin{Highlighting}[]
\KeywordTok{library}\NormalTok{(reshape2)}
\KeywordTok{head}\NormalTok{(airquality)}
\end{Highlighting}
\end{Shaded}

\begin{verbatim}
##   Ozone Solar.R Wind Temp Month Day
## 1    41     190  7.4   67     5   1
## 2    36     118  8.0   72     5   2
## 3    12     149 12.6   74     5   3
## 4    18     313 11.5   62     5   4
## 5    NA      NA 14.3   56     5   5
## 6    28      NA 14.9   66     5   6
\end{verbatim}

\begin{Shaded}
\begin{Highlighting}[]
\NormalTok{airqualityM<-}\KeywordTok{melt}\NormalTok{(airquality,}\DataTypeTok{id.vars =} \KeywordTok{c}\NormalTok{(}\StringTok{"Month"}\NormalTok{,}\StringTok{"Day"}\NormalTok{)) ##欄位需要保留"Month","Day"}
\KeywordTok{head}\NormalTok{(airqualityM)}
\end{Highlighting}
\end{Shaded}

\begin{verbatim}
##   Month Day variable value
## 1     5   1    Ozone    41
## 2     5   2    Ozone    36
## 3     5   3    Ozone    12
## 4     5   4    Ozone    18
## 5     5   5    Ozone    NA
## 6     5   6    Ozone    28
\end{verbatim}

轉換過的長表\texttt{airqualityM}資料框中,剩下Month, Day, variable,
value等四個欄位
(Column),屬於長表,以下範例variable欄位的值轉換為新欄位,並將value欄位填回新增的欄位,長表轉寬表範例如下:

\begin{Shaded}
\begin{Highlighting}[]
\KeywordTok{library}\NormalTok{(reshape2)}
\NormalTok{##欄位保留"Month","Day"外,其他欄位數目由variable定義}
\NormalTok{airqualityCast<-}\KeywordTok{dcast}\NormalTok{(airqualityM, Month +Day~variable) }
\KeywordTok{head}\NormalTok{(airqualityCast)}
\end{Highlighting}
\end{Shaded}

\begin{verbatim}
##   Month Day Ozone Solar.R Wind Temp
## 1     5   1    41     190  7.4   67
## 2     5   2    36     118  8.0   72
## 3     5   3    12     149 12.6   74
## 4     5   4    18     313 11.5   62
## 5     5   5    NA      NA 14.3   56
## 6     5   6    28      NA 14.9   66
\end{verbatim}

\section{遺漏值處理}

遺漏值(Missing
Value)常常出現在真實資料內,在數值運算時常會有問題,最簡單的方法是將有缺值的資料移除,如資料為向量,可使用\texttt{is.na()}來判斷資料是否為空值\texttt{NA},若為真\texttt{TRUE},則將資料移除。

\begin{Shaded}
\begin{Highlighting}[]
\NormalTok{naVec<-}\KeywordTok{c}\NormalTok{(}\StringTok{"a"}\NormalTok{,}\StringTok{"b"}\NormalTok{,}\OtherTok{NA}\NormalTok{,}\StringTok{"d"}\NormalTok{,}\StringTok{"e"}\NormalTok{)}
\KeywordTok{is.na}\NormalTok{(naVec)}
\end{Highlighting}
\end{Shaded}

\begin{verbatim}
## [1] FALSE FALSE  TRUE FALSE FALSE
\end{verbatim}

\begin{Shaded}
\begin{Highlighting}[]
\NormalTok{naVec[!}\KeywordTok{is.na}\NormalTok{(naVec)] ##保留所有在is.na()檢查回傳FALSE的元素}
\end{Highlighting}
\end{Shaded}

\begin{verbatim}
## [1] "a" "b" "d" "e"
\end{verbatim}

若資料型態為資料框,可使用\texttt{complete.cases}來選出完整的資料列,如果資料列是完整的,則會回傳真TRUE

\begin{Shaded}
\begin{Highlighting}[]
\KeywordTok{head}\NormalTok{(airquality)}
\end{Highlighting}
\end{Shaded}

\begin{verbatim}
##   Ozone Solar.R Wind Temp Month Day
## 1    41     190  7.4   67     5   1
## 2    36     118  8.0   72     5   2
## 3    12     149 12.6   74     5   3
## 4    18     313 11.5   62     5   4
## 5    NA      NA 14.3   56     5   5
## 6    28      NA 14.9   66     5   6
\end{verbatim}

\begin{Shaded}
\begin{Highlighting}[]
\KeywordTok{complete.cases}\NormalTok{(airquality) }
\end{Highlighting}
\end{Shaded}

\begin{verbatim}
##   [1]  TRUE  TRUE  TRUE  TRUE FALSE FALSE  TRUE  TRUE  TRUE FALSE FALSE  TRUE
##  [13]  TRUE  TRUE  TRUE  TRUE  TRUE  TRUE  TRUE  TRUE  TRUE  TRUE  TRUE  TRUE
##  [25] FALSE FALSE FALSE  TRUE  TRUE  TRUE  TRUE FALSE FALSE FALSE FALSE FALSE
##  [37] FALSE  TRUE FALSE  TRUE  TRUE FALSE FALSE  TRUE FALSE FALSE  TRUE  TRUE
##  [49]  TRUE  TRUE  TRUE FALSE FALSE FALSE FALSE FALSE FALSE FALSE FALSE FALSE
##  [61] FALSE  TRUE  TRUE  TRUE FALSE  TRUE  TRUE  TRUE  TRUE  TRUE  TRUE FALSE
##  [73]  TRUE  TRUE FALSE  TRUE  TRUE  TRUE  TRUE  TRUE  TRUE  TRUE FALSE FALSE
##  [85]  TRUE  TRUE  TRUE  TRUE  TRUE  TRUE  TRUE  TRUE  TRUE  TRUE  TRUE FALSE
##  [97] FALSE FALSE  TRUE  TRUE  TRUE FALSE FALSE  TRUE  TRUE  TRUE FALSE  TRUE
## [109]  TRUE  TRUE  TRUE  TRUE  TRUE  TRUE FALSE  TRUE  TRUE  TRUE FALSE  TRUE
## [121]  TRUE  TRUE  TRUE  TRUE  TRUE  TRUE  TRUE  TRUE  TRUE  TRUE  TRUE  TRUE
## [133]  TRUE  TRUE  TRUE  TRUE  TRUE  TRUE  TRUE  TRUE  TRUE  TRUE  TRUE  TRUE
## [145]  TRUE  TRUE  TRUE  TRUE  TRUE FALSE  TRUE  TRUE  TRUE
\end{verbatim}

\begin{Shaded}
\begin{Highlighting}[]
\KeywordTok{head}\NormalTok{(airquality[}\KeywordTok{complete.cases}\NormalTok{(airquality),]) ##保留所有在complete.cases()檢查回傳TRUE的元素}
\end{Highlighting}
\end{Shaded}

\begin{verbatim}
##   Ozone Solar.R Wind Temp Month Day
## 1    41     190  7.4   67     5   1
## 2    36     118  8.0   72     5   2
## 3    12     149 12.6   74     5   3
## 4    18     313 11.5   62     5   4
## 7    23     299  8.6   65     5   7
## 8    19      99 13.8   59     5   8
\end{verbatim}

利用演算法補值也是一種解決辦法,可參考\_skydome20\_的\href{http://www.rpubs.com/skydome20/R-Note10-Missing_Value}{R筆記--(10)遺漏值處理(Impute
Missing Value)}教學。

\chapter{探索式資料分析}\label{eda}

\section{資料內容初步分析統計}

\section{data.table}\label{data.table}

You can label chapter and section titles using \texttt{\{\#label\}}
after them, e.g., we can reference Chapter \ref{intro}. If you do not
manually label them, there will be automatic labels anyway, e.g.,
Chapter \ref{methods}.

Figures and tables with captions will be placed in \texttt{figure} and
\texttt{table} environments, respectively.

\begin{Shaded}
\begin{Highlighting}[]
\KeywordTok{par}\NormalTok{(}\DataTypeTok{mar =} \KeywordTok{c}\NormalTok{(}\DecValTok{4}\NormalTok{, }\DecValTok{4}\NormalTok{, .}\DecValTok{1}\NormalTok{, .}\DecValTok{1}\NormalTok{))}
\KeywordTok{plot}\NormalTok{(pressure, }\DataTypeTok{type =} \StringTok{'b'}\NormalTok{, }\DataTypeTok{pch =} \DecValTok{19}\NormalTok{)}
\end{Highlighting}
\end{Shaded}

\begin{figure}

{\centering \includegraphics[width=0.8\linewidth]{DataAnalyticsWithR_files/figure-latex/nice-fig6-1} 

}

\caption{Here is a nice figure!}\label{fig:nice-fig6}
\end{figure}

Reference a figure by its code chunk label with the \texttt{fig:}
prefix, e.g., see Figure \ref{fig:nice-fig}. Similarly, you can
reference tables generated from \texttt{knitr::kable()}, e.g., see Table
\ref{tab:nice-tab}.

\begin{Shaded}
\begin{Highlighting}[]
\NormalTok{knitr::}\KeywordTok{kable}\NormalTok{(}
  \KeywordTok{head}\NormalTok{(iris, }\DecValTok{20}\NormalTok{), }\DataTypeTok{caption =} \StringTok{'Here is a nice table!'}\NormalTok{,}
  \DataTypeTok{booktabs =} \OtherTok{TRUE}
\NormalTok{)}
\end{Highlighting}
\end{Shaded}

\begin{table}

\caption{\label{tab:nice-tab6}Here is a nice table!}
\centering
\begin{tabular}[t]{rrrrl}
\toprule
Sepal.Length & Sepal.Width & Petal.Length & Petal.Width & Species\\
\midrule
5.1 & 3.5 & 1.4 & 0.2 & setosa\\
4.9 & 3.0 & 1.4 & 0.2 & setosa\\
4.7 & 3.2 & 1.3 & 0.2 & setosa\\
4.6 & 3.1 & 1.5 & 0.2 & setosa\\
5.0 & 3.6 & 1.4 & 0.2 & setosa\\
\addlinespace
5.4 & 3.9 & 1.7 & 0.4 & setosa\\
4.6 & 3.4 & 1.4 & 0.3 & setosa\\
5.0 & 3.4 & 1.5 & 0.2 & setosa\\
4.4 & 2.9 & 1.4 & 0.2 & setosa\\
4.9 & 3.1 & 1.5 & 0.1 & setosa\\
\addlinespace
5.4 & 3.7 & 1.5 & 0.2 & setosa\\
4.8 & 3.4 & 1.6 & 0.2 & setosa\\
4.8 & 3.0 & 1.4 & 0.1 & setosa\\
4.3 & 3.0 & 1.1 & 0.1 & setosa\\
5.8 & 4.0 & 1.2 & 0.2 & setosa\\
\addlinespace
5.7 & 4.4 & 1.5 & 0.4 & setosa\\
5.4 & 3.9 & 1.3 & 0.4 & setosa\\
5.1 & 3.5 & 1.4 & 0.3 & setosa\\
5.7 & 3.8 & 1.7 & 0.3 & setosa\\
5.1 & 3.8 & 1.5 & 0.3 & setosa\\
\bottomrule
\end{tabular}
\end{table}

You can write citations, too. For example, we are using the
\textbf{bookdown} package \citep{R-bookdown} in this sample book, which
was built on top of R Markdown and \textbf{knitr} \citep{xie2015}.

\chapter{資料視覺化}\label{vis}

You can label chapter and section titles using \texttt{\{\#label\}}
after them, e.g., we can reference Chapter \ref{intro}. If you do not
manually label them, there will be automatic labels anyway, e.g.,
Chapter \ref{methods}.

Figures and tables with captions will be placed in \texttt{figure} and
\texttt{table} environments, respectively.

\begin{Shaded}
\begin{Highlighting}[]
\KeywordTok{par}\NormalTok{(}\DataTypeTok{mar =} \KeywordTok{c}\NormalTok{(}\DecValTok{4}\NormalTok{, }\DecValTok{4}\NormalTok{, .}\DecValTok{1}\NormalTok{, .}\DecValTok{1}\NormalTok{))}
\KeywordTok{plot}\NormalTok{(pressure, }\DataTypeTok{type =} \StringTok{'b'}\NormalTok{, }\DataTypeTok{pch =} \DecValTok{19}\NormalTok{)}
\end{Highlighting}
\end{Shaded}

\begin{figure}

{\centering \includegraphics[width=0.8\linewidth]{DataAnalyticsWithR_files/figure-latex/nice-fig7-1} 

}

\caption{Here is a nice figure!}\label{fig:nice-fig7}
\end{figure}

Reference a figure by its code chunk label with the \texttt{fig:}
prefix, e.g., see Figure \ref{fig:nice-fig}. Similarly, you can
reference tables generated from \texttt{knitr::kable()}, e.g., see Table
\ref{tab:nice-tab}.

\begin{Shaded}
\begin{Highlighting}[]
\NormalTok{knitr::}\KeywordTok{kable}\NormalTok{(}
  \KeywordTok{head}\NormalTok{(iris, }\DecValTok{20}\NormalTok{), }\DataTypeTok{caption =} \StringTok{'Here is a nice table!'}\NormalTok{,}
  \DataTypeTok{booktabs =} \OtherTok{TRUE}
\NormalTok{)}
\end{Highlighting}
\end{Shaded}

\begin{table}

\caption{\label{tab:nice-tab7}Here is a nice table!}
\centering
\begin{tabular}[t]{rrrrl}
\toprule
Sepal.Length & Sepal.Width & Petal.Length & Petal.Width & Species\\
\midrule
5.1 & 3.5 & 1.4 & 0.2 & setosa\\
4.9 & 3.0 & 1.4 & 0.2 & setosa\\
4.7 & 3.2 & 1.3 & 0.2 & setosa\\
4.6 & 3.1 & 1.5 & 0.2 & setosa\\
5.0 & 3.6 & 1.4 & 0.2 & setosa\\
\addlinespace
5.4 & 3.9 & 1.7 & 0.4 & setosa\\
4.6 & 3.4 & 1.4 & 0.3 & setosa\\
5.0 & 3.4 & 1.5 & 0.2 & setosa\\
4.4 & 2.9 & 1.4 & 0.2 & setosa\\
4.9 & 3.1 & 1.5 & 0.1 & setosa\\
\addlinespace
5.4 & 3.7 & 1.5 & 0.2 & setosa\\
4.8 & 3.4 & 1.6 & 0.2 & setosa\\
4.8 & 3.0 & 1.4 & 0.1 & setosa\\
4.3 & 3.0 & 1.1 & 0.1 & setosa\\
5.8 & 4.0 & 1.2 & 0.2 & setosa\\
\addlinespace
5.7 & 4.4 & 1.5 & 0.4 & setosa\\
5.4 & 3.9 & 1.3 & 0.4 & setosa\\
5.1 & 3.5 & 1.4 & 0.3 & setosa\\
5.7 & 3.8 & 1.7 & 0.3 & setosa\\
5.1 & 3.8 & 1.5 & 0.3 & setosa\\
\bottomrule
\end{tabular}
\end{table}

You can write citations, too. For example, we are using the
\textbf{bookdown} package \citep{R-bookdown} in this sample book, which
was built on top of R Markdown and \textbf{knitr} \citep{xie2015}.

\chapter{從小數據到大數據分析}\label{big}

\section{R + Hadoop}\label{r-hadoop}

\section{RHadoop安裝測試流程 (Cloudera)}\label{rhadoop-cloudera}

安裝與測試日期2016/05/12

\subsection{系統/軟體版本資訊}

\begin{itemize}
\tightlist
\item
  Cloudera Hadoop Platform: CDH-5.4.5
  \href{http://www.cloudera.com/downloads/cdh/5-4-5.html}{下載}
\item
  R for Linux 3.3.0
  \href{https://cran.rstudio.com/bin/linux/redhat/README}{安裝說明}
\item
  RStudio Server
  \href{https://www.rstudio.com/products/rstudio/download-server/}{下載}
\item
  RHadoop (latest version on May 12, 2016)
  \href{https://github.com/RevolutionAnalytics/RHadoop/wiki/Downloads}{下載}

  \begin{itemize}
  \tightlist
  \item
    ravro-1.0.3
  \item
    plyrmr-0.6.0
  \item
    rmr-3.3.1
  \item
    rhdfs-1.0.8
  \item
    rhbase-1.2.1
  \end{itemize}
\end{itemize}

\subsection{參考資料}

\begin{itemize}
\tightlist
\item
  \href{https://github.com/RevolutionAnalytics/RHadoop/wiki/Installing-RHadoop-on-RHEL}{RHadoop安裝說明文件}
\item
  \href{https://bigdatastudy.hackpad.com/ep/pad/static/IADMBeqF0vV}{RHadoop安裝步驟}
\item
  \href{http://unix.stackexchange.com/questions/271514/setting-persistent-environment-variable-in-centos-7-issue}{Setting
  persistent environment variable in CentOS 7 issue}
\item
  \href{https://community.cloudera.com/t5/CDH-Manual-Installation/How-to-resolve-quot-Permission-denied-quot-errors-in-CDH/ta-p/36141}{How
  to resolve ``Permission denied'' errors in CDH}
\end{itemize}

\subsection{安裝步驟}

\begin{enumerate}
\def\labelenumi{\arabic{enumi}.}
\tightlist
\item
  下載Cloudera CDH QuickStart VM
  \href{http://www.cloudera.com/developers/get-started-with-hadoop-tutorial.html}{Cloudera
  VM}
\item
  安裝R
  \href{https://cran.rstudio.com/bin/linux/redhat/README}{安裝說明}
\item
  安裝RHadoop
  \href{https://bigdatastudy.hackpad.com/ep/pad/static/IADMBeqF0vV}{RHadoop安裝步驟}
\item
  安裝RStudio Server
  \href{https://www.rstudio.com/products/rstudio/download-server/}{說明}
\end{enumerate}

\subsubsection{Cloudera CDH QuickStart
VM}\label{cloudera-cdh-quickstart-vm}

Cloudera CDH QuickStart
VM是由Cloudera提供的虛擬機器,內涵Linux系統與預載多項Hadoop相關服務,適合想了解Hadoop運作的初學者。

下載VM後,用Virtural Box 開啟即可。

\begin{itemize}
\tightlist
\item
  \href{http://www.cloudera.com/developers/get-started-with-hadoop-tutorial.html}{Cloudera
  CDH QuickStart VM下載處}
\item
  \href{https://www.virtualbox.org/}{Virtural Box下載處}
\end{itemize}

以下安裝步驟都在Cloudera CDH QuickStart VM內進行

\subsubsection{安裝R}\label{r}

\begin{itemize}
\tightlist
\item
  Cloudera CDH用的Linux作業系統是CentOS
\item
  依照安裝說明,需要先安裝Extra Packages for Enterprise Linux
  (EPEL),但系統內有預載,所以可以不用按照說明重新下載安裝,直接執行\texttt{sudo\ yum\ install\ epel-release}指令即可
\item
  步驟:安裝最新EPRL,更新yum,安裝R。打開Terminal輸入以下指令。
\end{itemize}

\begin{verbatim}
sudo yum install epel-release
sudo yum update
sudo yum install R
\end{verbatim}

\subsubsection{安裝RHadoop-1 先進行環境設定}\label{rhadoop-1-}

設定\texttt{HADOOP\_CMD}與\texttt{HADOOP\_STREAMING}兩項環境參數,路徑可能會不同(尤其是\texttt{HADOOP\_STREAMING})

\begin{enumerate}
\def\labelenumi{\arabic{enumi}.}
\item
  尋找\texttt{HADOOP\_STREAMING}路徑方法

\begin{verbatim}
find / -name hadoop-streaming-*.jar
\end{verbatim}
\item
  設定\texttt{HADOOP\_CMD}與\texttt{HADOOP\_STREAMING}兩項環境參數,路徑記得換成自己的

\begin{verbatim}
echo export HADOOP_CMD="/usr/bin/hadoop">/etc/profile.d/hadoopenv.sh
echo export HADOOP_STREAMING=
"/opt/cloudera/parcels/CDH-5.4.5-1.cdh5.4.5.p0.7/lib/hadoop-mapreduce/
    hadoop-streaming-2.6.0-cdh5.4.5.jar" > /etc/profile.d/hadoopenv.sh
chmod 0755 /etc/profile.d/hadoopenv.sh
\end{verbatim}
\end{enumerate}

\subsubsection{安裝RHadoop-2 rmr2}\label{rhadoop-2-rmr2}

\begin{itemize}
\tightlist
\item
  每個Node都要裝
\item
  安裝前先至\href{https://github.com/RevolutionAnalytics/rmr2/blob/master/pkg/DESCRIPTION}{說明檔}看需要先安裝哪些其他的packages,Depends
  和 Imports 所列的packages都要裝
\item
  以下為安裝packages的程式碼,在R內執行(在Terminal輸入\texttt{R},就能進入R軟體)
\end{itemize}

\begin{Shaded}
\begin{Highlighting}[]
\KeywordTok{install.packages}\NormalTok{(}\KeywordTok{c}\NormalTok{(}\StringTok{"methods"}\NormalTok{,}\StringTok{"Rcpp"}\NormalTok{, }\StringTok{"RJSONIO"}\NormalTok{, }\StringTok{"digest"}\NormalTok{, }\StringTok{"functional"}\NormalTok{, }
                   \StringTok{"reshape2"}\NormalTok{,}\StringTok{"stringr"}\NormalTok{, }\StringTok{"plyr"}\NormalTok{, }\StringTok{"caTools"}\NormalTok{,}\StringTok{"quickcheck"}\NormalTok{,}\StringTok{"testthat"}\NormalTok{), }
                 \DataTypeTok{dependencies=}\OtherTok{TRUE}\NormalTok{, }\DataTypeTok{repos=}\StringTok{'http://cran.rstudio.com/'}\NormalTok{)}
\end{Highlighting}
\end{Shaded}

\begin{itemize}
\tightlist
\item
  使用\texttt{q()}指令,跳出R軟體
\item
  \href{https://github.com/RevolutionAnalytics/RHadoop/wiki/Downloads}{下載rmr2}
\item
  安裝(請將\texttt{rmr2\_2.3.0.tar.gz}替換成剛剛下載的安裝檔路徑)
\end{itemize}

\begin{verbatim}
sudo R CMD INSTALL rmr2_2.3.0.tar.gz
\end{verbatim}

\subsubsection{安裝RHadoop-3 rhdfs}\label{rhadoop-3-rhdfs}

\begin{itemize}
\tightlist
\item
  只要裝在會跑R的那個Node
\item
  在裝之前,先Check是否有安裝JDK (測試JDK 1.8.0\_91沒問題)
\item
  Check環境變數JAVA\_HOME是否有設好
\end{itemize}

\begin{verbatim}
echo $JAVA_HOME
\end{verbatim}

若什麼都沒有回傳,先設定環境變數(將\texttt{/usr/java/jdk1.8.0\_91}換成自己的路徑)

\begin{verbatim}
echo export JAVA_HOME="/usr/java/jdk1.8.0_91">/etc/profile.d/jdkenv.sh
\end{verbatim}

為了讓R可以跑JAVA,在Terminal輸入

\begin{verbatim}
R CMD javareconf
\end{verbatim}

然後進到R程式(在Terminal輸入\texttt{R},就能進入R軟體),安裝\texttt{rJava}
package

\begin{Shaded}
\begin{Highlighting}[]
\KeywordTok{install.packages}\NormalTok{(}\StringTok{"rJava"}\NormalTok{,}\DataTypeTok{dependencies=}\OtherTok{TRUE}\NormalTok{, }\DataTypeTok{repos=}\StringTok{'http://cran.rstudio.com/'}\NormalTok{)}
\end{Highlighting}
\end{Shaded}

最後跳出R程式,\href{https://github.com/RevolutionAnalytics/RHadoop/wiki/Downloads}{下載rhdfs},安裝rhdfs

\begin{itemize}
\tightlist
\item
  將\texttt{/usr/bin/hadoop}換成自己的\texttt{HADOOP\_CMD}路徑
\item
  \texttt{rhdfs\_1.0.8.tar.gz}換成下載的安裝檔路徑)
\end{itemize}

\begin{verbatim}
sudo HADOOP_CMD=/usr/bin/hadoop R CMD INSTALL rhdfs_1.0.8.tar.gz
\end{verbatim}

\subsection{測試前,先解決權限問題}

\begin{itemize}
\tightlist
\item
  預設hdfs的存取權限不足,所以要打開
\item
  將\texttt{user01}改為自己的使用者名稱
\end{itemize}

\begin{verbatim}
sudo -u hdfs hadoop fs -mkdir /user/user01
sudo -u hdfs hadoop fs -chown user01 /user/user01
\end{verbatim}

\subsection{測試}

進入R測試以下程式碼是否能執行

\begin{Shaded}
\begin{Highlighting}[]
\KeywordTok{Sys.setenv}\NormalTok{(}\DataTypeTok{HADOOP_CMD=}\StringTok{"/usr/bin/hadoop"}\NormalTok{)}
\KeywordTok{Sys.setenv}\NormalTok{(}\DataTypeTok{HADOOP_STREAMING=}\StringTok{"/opt/cloudera/parcels/CDH-5.4.5-1.cdh5.4.5.p0.7/lib/hadoop-mapreduce/hadoop-streaming-2.6.0-cdh5.4.5.jar"}\NormalTok{)}
\KeywordTok{library}\NormalTok{(rmr2)}
\CommentTok{#test mapreduce}
\NormalTok{small.ints =}\StringTok{ }\KeywordTok{to.dfs}\NormalTok{(}\DecValTok{1}\NormalTok{:}\DecValTok{100}\NormalTok{)}
\NormalTok{out<-}\KeywordTok{mapreduce}\NormalTok{(}
    \DataTypeTok{input =} \NormalTok{small.ints, }
    \DataTypeTok{map =} \NormalTok{function(., v) }\KeywordTok{cbind}\NormalTok{(v, v^}\DecValTok{2}\NormalTok{))}
\KeywordTok{head}\NormalTok{(}\KeywordTok{from.dfs}\NormalTok{(out))}
\end{Highlighting}
\end{Shaded}

\subsection{安裝RStudio Server}\label{rstudio-server}

\href{https://www.rstudio.com/products/rstudio/download-server/}{官方下載與安裝說明}

在Terminal執行以下程式碼

\begin{itemize}
\tightlist
\item
  檔案連結\texttt{https://download2.rstudio.org/rstudio-server-rhel-0.99.896-x86\_64.rpm}可能有最新版,請Check\href{https://www.rstudio.com/products/rstudio/download-server/}{官網}
\end{itemize}

\begin{verbatim}
wget https://download2.rstudio.org/rstudio-server-rhel-0.99.896-x86_64.rpm
sudo yum install --nogpgcheck rstudio-server-rhel-0.99.896-x86_64.rpm
\end{verbatim}

打開瀏覽器,輸入\texttt{http://localhost:8787/},就能進入RStudio
Server了!

測完收工~

\section{RHadoop MapReduce: easy word
count}\label{rhadoop-mapreduce-easy-word-count}

\begin{Shaded}
\begin{Highlighting}[]
\NormalTok{Debate<-}\KeywordTok{readLines}\NormalTok{(}\StringTok{"https://raw.githubusercontent.com/yijutseng/BigDataCGUIM/master/RepDebateMiami.txt"}\NormalTok{)}
\NormalTok{DebateSplit<-}\KeywordTok{unlist}\NormalTok{(}\KeywordTok{strsplit}\NormalTok{(}\KeywordTok{tolower}\NormalTok{(Debate),}\DataTypeTok{split =} \StringTok{' |}\CharTok{\textbackslash{}\textbackslash{}}\StringTok{.|}\CharTok{\textbackslash{}\textbackslash{}}\StringTok{,|}\CharTok{\textbackslash{}\textbackslash{}}\StringTok{?'}\NormalTok{))}
\CommentTok{#table(DebateSplit)}
\end{Highlighting}
\end{Shaded}

\begin{Shaded}
\begin{Highlighting}[]
\NormalTok{DebateSplitDFS =}\StringTok{ }\KeywordTok{to.dfs}\NormalTok{(DebateSplit)}
\NormalTok{result =}\StringTok{ }\KeywordTok{mapreduce}\NormalTok{(}
    \DataTypeTok{input =} \NormalTok{DebateSplitDFS,}
    \DataTypeTok{map =} \NormalTok{function(.,v) }\KeywordTok{keyval}\NormalTok{(v, }\DecValTok{1}\NormalTok{),}
    \DataTypeTok{reduce =} \NormalTok{function(k,vv) }\KeywordTok{keyval}\NormalTok{(k, }\KeywordTok{sum}\NormalTok{(vv)))}
\KeywordTok{head}\NormalTok{(result)}
\end{Highlighting}
\end{Shaded}

\section{R + Spark}\label{r-spark}

\chapter{軟體安裝}\label{install}

\section{R}\label{r-2}

\section{RStudio}\label{rstudio}

\chapter*{作者資訊}\label{author}
\addcontentsline{toc}{chapter}{作者資訊}

曾意儒 Yi-Ju Tseng

\href{http://im.cgu.edu.tw/bin/home.php}{長庚大學 資訊管理學系} 助理教授
\href{http://yijutseng.github.io}{個人網站} Lab:
\href{http://yijutseng.github.io/Lab/}{數位健康實驗室}

\bibliography{packages.bib,book.bib}


\end{document}
